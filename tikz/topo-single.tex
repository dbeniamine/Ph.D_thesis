%!TEX encoding=UTF-8 Unicode
%Palette GnBu 5 col + white
\definecolor{ColPU}{HTML}{FFFFFF}
\definecolor{ColCore}{HTML}{F0F9E8}
\definecolor{ColL1}{HTML}{BAE4BC}
\definecolor{ColL2}{HTML}{7BCCC4}
\definecolor{ColL3}{HTML}{43A2CA}
\definecolor{ColM}{HTML}{0868AC}
\definecolor{ColS}{HTML}{FFFFFF}

\pgfdeclarelayer{bg}
\pgfdeclarelayer{bbg}
\pgfdeclarelayer{bbbg}
\pgfsetlayers{bbbg,bbg,bg,main}


\tikzset{
    box/.style={
        shape=rectangle,
        text centered,
        draw,
    },
    pics/core/.style args={#1#2#3#4}{
        % Args: #1: nb PU, #2 core id, #3 direction: + or -
        code={
            \pgfmathtruncatemacro{\pmin}{#1*#2}
            \pgfmathtruncatemacro{\pmax}{\pmin+#1-1}
            %PUs
            \foreach \pu in {\pmin,...,\pmax}{
                \pgfmathtruncatemacro{\pstep}{\pu-\pmin}
                \node[box,fill=ColPU] (PU-\pu)at (0,#3.7*\pstep) {PU\#\pu};
            }
            %Core ID
            \node (inv) at ($(PU-\pmin)#3(0,-.3)$) {};
            \node[minimum width=3.3em] (name) at ($(PU-\pmin)!0.5!(PU-\pmax)#3(0,1)$) {Core\##2};
            % L1
            \begin{pgfonlayer}{bg}
                \node[box,fill=ColCore,inner sep=.1pt, fit=(name) (PU-\pmin)
                (PU-\pmax) (inv)] (core-#2) {};
            \end{pgfonlayer}
            \draw[fill=ColL1] ($(core-#2.#4 west)#3(0,.1)$) rectangle ($(core-#2.#4 east)#3(0,.5)$)%
            node[pos=.5] (l1) {L1};
            % links
            \draw (core-#2.#4) -- ($(core-#2.#4)#3(0,.1)$);
            \coordinate (l1-#2-n) at ($(core-#2.#4)#3(0,.5)$);
        }
    },
    pics/l2group/.style args={#1#2#3#4}{
        % Args: #1: nb Cores, #2 group id, #3 direction: + or -
        code={
            \pgfmathtruncatemacro{\cmin}{#1*#2}
            \pgfmathtruncatemacro{\cmax}{\cmin+#1-1}
            % Cores
            \foreach \core in {\cmin,...,\cmax}{
                \pgfmathtruncatemacro{\cstep}{\core-\cmin}
                \draw (1.7*\cstep,0) pic {core={2}{\core}{#3}{#4}};
            }
            % L2
            \draw[fill=ColL2] ($(core-\cmin.#4 west)#3(0,.6)$) rectangle
            ($(core-\cmax.#4 east)#3(0,1.1)$) node[pos=.5]{L2};
            % Coordinates for L3
            \coordinate (l2g-#2-w) at ($(core-\cmin.#4 west)#3(0,1.1)$);
            \coordinate (l2g-#2-e) at ($(core-\cmax.#4 east)#3(0,1.1)$);
            %links
            \foreach \core in {\cmin,...,\cmax}{
                \draw (l1-\core-n) -- ($(core-\core.#4)#3(0,.6)$);
            }
        }
    },
    pics/socket/.style args={#1#2#3#4}{
        % Args: #1: nb l2 groups, #2 socket id, #3 arguments for underlying pics
        code={
            \pgfmathtruncatemacro{\lmin}{#1*#2}
            \pgfmathtruncatemacro{\lmax}{\lmin+#1-1}
            % Cores
            \foreach \lg in {\lmin,...,\lmax}{
                \pgfmathtruncatemacro{\lgstep}{\lg-\lmin}
                \draw (3.4*\lgstep,0) pic {l2group={2}{\lg}{#3}{#4}};
            }
           % L3
            \draw[fill=ColL3] ($(l2g-\lmin-w)#3(0,.1)$) rectangle ($(l2g-\lmax-e)#3(0,.5)$)%
            node[pos=.5]{L3};
           % Coordinates for Mem
            \coordinate (s-#2-w) at ($(l2g-\lmin-w)#3(0,.5)$);
            \coordinate (s-#2-e) at ($(l2g-\lmax-e)#3(0,.5)$);
           % links
           \foreach \lg in {\lmin,...,\lmax}{
               \draw ($(l2g-\lg-w)!.5!(l2g-\lg-e)$) --
                   ($(l2g-\lg-w)!.5!(l2g-\lg-e)#3(0,.1)$);
           }
           % CPU
            \ifthenelse{\equal{#3}{+}}{
                \node (minnode) at (-.6,-.3) {};
            }{
                \node (minnode) at (-.6,.3) {};
            }
            \node (maxnode) at ($(l2g-\lmax-e)#3(0,1)$) {};
            \node (sockname) at ($(l2g-\lmin-e)#3(0,.8)$) {CPU};

            \begin{pgfonlayer}{bbg}
                \node[box,fill=ColS,fit=(minnode) (maxnode)] (cpu-#2) {};
            \end{pgfonlayer}
           % Memory
            \draw[fill=ColM,text=white] ($(cpu-#2.#4 west)#3(0,.5)$) rectangle ($(cpu-#2.#4 east)#3(0,1.5)$)%
            node[pos=.5]{Memory};

            \draw[line width=.5em] (cpu-#2.#4) -- ($(cpu-#2.#4)#3(0,.5)$);

        }
    },
}

\begin{tikzpicture}[font=\small, every pic/.style={scale=.9}]
    \pic at (0,0)  {socket={2}{0}{+}{north}};
\end{tikzpicture}
% vim: et si sta lbr  sw=4 ts=4 spelllang=en_us
