%!TEX encoding=UTF-8 Unicode
\documentclass[a4paper,10pt]{book}
\usepackage[a4paper]{style/meta-donnees}

%========================= Encodage fontes et langues =========================

\usepackage[utf8]{inputenc}
\usepackage[french,english]{babel}
\usepackage[T1]{fontenc}

%========================== Fancy =============================================
\usepackage{etoolbox}

\newif\ifpartialtoc

\usepackage{titlesec}
\usepackage{titletoc}
\usepackage{shorttoc}

% pages style
\newpagestyle{main}{
    \setheadrule{.55pt}
    \sethead[\thepage][][\Roman{chapter}\ --\ \chaptertitle] % even
    {\thesection\ --\ \sectiontitle}{}{\thepage} %odd
}


% Chapter in toc
 \titlecontents*{chapter}% <section-type>
   [0pt]% <left>
   {}% <above-code>
   {\bfseries \uppercase\expandafter{\romannumeral\thecontentslabel\relax}\quad}% <numbered-entry-format>
   {}% <numberless-entry-format>
   {\bfseries\hfill\contentspage}% <filler-page-format>

% chapter pages
\titleformat{\chapter}[display]
    {\bfseries\huge}
    {\filleft\chaptertitlename\Huge\ \Roman{chapter}}
    {4ex}
    {\titleline{\titlerule[.25ex]}
        \vspace{2ex}%
    }
    [\vspace{2ex}%
    \titleline{\titlerule[.25ex]}%
    \ifpartialtoc
    \vspace*{2ex}%
    \startcontents
    \begin{minipage}{.9\linewidth}
        \filcenter \Large\contentsname\\
        \vspace*{1ex}%
        \titlerule%
        \vspace*{1ex}%
        \printcontents{}{1}{\large}
        \vspace*{1ex}%
        \titlerule%
    \end{minipage}
    \fi
    \vspace*{4ex}%
    ]



%========================= Mise en forme de code ==============================
\usepackage{algorithm}
\usepackage{algpseudocode}

\algblockdefx[]{Function}{EndFunction}
[2]{\algorithmicfunction\ \textproc{#1}{(#2)}}%
{\algorithmicend\ \algorithmicfunction}%


\algnewcommand\Callp[2]{\textproc{#1}(#2)}%

%========================== Liens =============================================

% gestion des hyperliens
\usepackage{hyperref}
\hypersetup{
    colorlinks=true, %colore les liens
    breaklinks=true, %permet le retour à la ligne dans les liens trop longs
    urlcolor= blue, %couleur des hyperliens
    linkcolor= black, %couleur des liens internes
    bookmarksopen=true,
    citecolor=black,
}
\usepackage{url}

%========================== Images ============================================

\usepackage[]{caption}
\usepackage[]{subcaption}
\usepackage{graphicx}
\graphicspath{{./img/}{./style/}}

%========================== Tikz ==============================================

\usepackage{tikz}
%% TODO

%========================== Todonotes =========================================

%\usepackage[obeyDraft]{todonotes}
\usepackage{todonotes}
\newcommand{\DB}[1]{\todo[author=David,inline]{#1}}
\newcommand{\GH}[1]{\todo[author=Guillaume,inline]{#1}}


%========================== CustomCommands ====================================

\newcommand{\Input}[1]{\input{tex/#1}}

%========================== Misc ==============================================

\usepackage{siunitx}
\usepackage{multirow,booktabs}

\begin{document}

%========================== Style =============================================
%%%%%%%%%%%%%%%%%%%%%%%%%%%%%%%%%%%%%%%%%%%%%%%%%%%%%%
%%             Commandes Meta-données               %%
%%   à renseigner par les auteurs pour générer      %%
%%     la couverture modèle Univ. Grenoble          %%
%%%%%%%%%%%%%%%%%%%%%%%%%%%%%%%%%%%%%%%%%%%%%%%%%%%%%%
%%      Fichier encodé au format ISO-8859-16        %%

%\Sethpageshift{???mm}   %%optionnel : à décommenter si besoin pour ajout d'espace afin de center la couvérture horizontalement (valeur par défaut est -5.5mm)
%\Setvpageshift{???mm}   %%optionnel : à décommenter si besoin pour ajout d'espace afin de center la couvérture verticalement (valeur par défaut est -15.5mm)


%\Universite{}    %%optionnel : à décommenter et à renseigenr si vous voulez changer le non d'université
%\Grade{}         %%optionnel : à décommenter et à renseigenr si vous voulez changer le grade
\Specialite{Informatique}
\Arrete{}
\Auteur{David Beniamine}
\Directeur{Bruno Raffin}
\CoDirecteur{Guillaume Huard}    %%optionnel : à décommenter et à renseigenr si présence d'un Co-directeur de thèse
\Laboratoire{LIG}
\EcoleDoctorale{EDMSTII}         
\Titre{Something about performance analysis and stuff}
%\Soustitre{}      %%optionnel : à décommenter et à renseigenr si présence d'un sous-titre de thèse
\Depot{}       


% Commande pour création de nouvelles catégories dans le jury:

%\UGTNewJuryCategory{...NomDeLaCategorie...}{...Definition...}

% Exemple \UGTNewJuryCategory{UGTFamille}{Membre de la famille} que nous ajoutons dans la commande \Jury ci-dessous sous la forme \UGTFamille{Jean Rousseau}{(...titre_et_affiliation...s'il_y_en_a...)}


\Jury{
    \UGTPresident{Pr, Jes\'us Labarta Mancho}{Professor at Universitat Polit\`ecnica de Catalunya}
%\UGTPresidente{...Civilité, Prénom\_et\_Nom...}{...titre\_et\_affiliation...}

\UGTRapporteur{Pr, Martin Quinson}{Professor at ENS Rennes}      %% 1er rapporteur
\UGTRapporteur{Pr,  Raymond Namyste}{Professor at University of Bordeaux}      %% second rapporteur
 
\UGTExaminateur{Dr, Lucas M. Schnorr}{Associate professor at Institute of Informatics of the Federal University of Rio Grande do Sul}     %% 1er examinateur
%\UGTExaminateur{...Civilité, Prénom\_et\_Nom...}{...titre\_et\_affiliation...}     %% second examinateur
%\UGTExaminatrice{...Civilité, Prénom\_et\_Nom...}{...titre\_et\_affiliation...}    %% 3ème examinateur

\UGTDirecteur{Pr,Bruno Raffin}{Professeur at INRIA Grenoble}       %% Directeur de thèse
\UGTCoDirecteur{Dr, Guillaume Huard}{Assistant professor at Grenoble Alpes University}     %% Co-Directeur de thèse s'il y en a

%\UGTInvite{...Civilité, Prénom\_et\_Nom...}{...titre\_et\_affiliation...}
%\UGTInvitee{...Civilité, Prénom\_et\_Nom...}{...titre\_et\_affiliation...}
}

\frontmatter
\pagestyle{empty}
\MakeUGthesePDG


\shorttableofcontents{Outline}{1}


\chapter*{Abstract}
    \DB{abstract in French and English}

\mainmatter
\partialtoctrue
\pagestyle{main}

\Input{intro}
\Input{profiling}
\Input{tracing}
\Input{analyzing}
\Input{cncl}

\backmatter
\partialtocfalse

\bibliographystyle{apalike}
\bibliography{biblio}

\tableofcontents
\listoffigures
\listoftables

\chapter*{Acknowledgments}
\DB{C'est là quj'ai le droit de dire plein de conneries !}

%==============================================================================

\end{document}
