%!TEX encoding=UTF-8 Unicode
\usepackage[acronym,nomain,toc]{glossaries}

\makeglossaries

%
% Accronyms: {{{1
%
%   \newacronym{name}{SIGLE}{signification}

\newacronym{API}{API}{Application Programming Interface}

\newacronym{HPC}{HPC}{High Performance Computing}

\newacronym{npb}{NPB}{Nas Parallel Benchmarks}

\newacronym{CPU}{CPU}{Central Processing Unit}

\newacronym{GPGPU}{GPGPU}{General Purpose Graphical Processing Unit}

\newacronym{GPU}{GPU}{Graphical Processing Unit}

\newacronym{VCS}{VCS}{Version Control System}

\newacronym{SOFA}{SOFA}{Simulation Open Framework Architecture}

\newacronym{PEBS}{PEBS}{Precise Event Based Sampling}

\newacronym{IBS}{IBS}{Instruction Based Sampling}

\newacronym{OS}{OS}{Operating System}

\newacronym{NUMA}{NUMA}{Non-Uniform Memory Access}

\newacronym{Moca}{Moca}{Memory Organisation Carthography \& Analysis}

\newacronym{Tabarnac}{Tabarnac}{Tools for Analyzing the Behavior of Applications Running on NUMA ArChitecture}

\newacronym{PAPI}{PAPI}{Performances API}

\newacronym{Likwid}{Likwid}{Like I Knew What I am Doing}

\newacronym{PCM}{PCM}{Performance Counter Monitor}

\newacronym{MPI}{MPI}{Message Passing Interface}

\newacronym{OpenMP}{OpenMP}{Open Multi-Processing}

\newacronym{MAQAO}{MAQAO}{Modular Assembler Quality Analyzer and Optimizer}

\newacronym{TAU}{TAU}{Tunning and Analysis Utilities}

\newacronym{PARAVER}{PARAVER}{PARAllel Visualization end Events Representation}

\newacronym{I/O}{I/O}{Input / Output}

\newacronym{BLAS}{BLAS}{Basic Linear Algebra Subprograms}

\newacronym{LAPACK}{LAPACK}{Linear Algebra PACKage}

%
% Glossary entries (technical words): {{{1
%
% \newglossaryentry{label}{
%   name=name,
%   description={some text}
%   }

\newglossaryentry{Pin}{name=Pin,
    description={Pin is a dynamic binary instrumentation framework designed by
    Intel for IA-32 and X86-64 architectures}}

\newglossaryentry{Pintool}{name=Pintool,
    description={Pintool is the nave given (by Intel) to tools developed using
    Pin}}


\newglossaryentry{Linux}{name=Linux,
    description={The Linux kernel}}

\newglossaryentry{Debian}{name=Debian,
    description={Debian is a free GNU/Linux distribution}}

\newglossaryentry{Perf}{name=Perf,
    description={The Linux perf command, also called perf\_events, allows to
    access performance counters}}

\newglossaryentry{OProfile}{name=Oprofile,
    description={OProfile is an open source project that includes a
    statistical profiler for Linux systems}}

\newglossaryentry{PerfSuite}{name=PerfSuite,
    description={A collection of performance analysis software}}

\newglossaryentry{HPCToolkit}{name=HPCToolkit,
    description={HPCToolkit is an integrated suite of tools for measurement
    and analysis of program performance}}

\newglossaryentry{CodeAnalyst}{name=CodeAnalyst,
    description={CodeAnalyst is AMD performance analyser}}

\newglossaryentry{CodeXL}{name=CodeXL,
    description={CodeXL is CodeAnalyst successor's}}

\newglossaryentry{AMD}{name=AMD,
    description={Advanced Micro Devices, Inc}}

\newglossaryentry{Intel}{name=Intel,
    description={Intel corporation}}

\newglossaryentry{VTune}{name=VTune,
    description={Intel VTune Amplifier}}

\newglossaryentry{KAAPI}{name=KAAPI,
    description={KAAPI is a parallel runtime with data flow dependencies}}

\newglossaryentry{SimGrid}{name=SimGrid,
    description={SimGrid is a scientific instrument to study the behavior of
        large-scale distributed systems such as Grids, Clouds, HPC or P2P
    systems}}

\newglossaryentry{Framesoc}{name=Framesoc,
    description={Framesoc is a generic trace management and analysis infrastructure}}

\newglossaryentry{Git}{name=Git,
    description={Git is a distributed Version Control System}}

\newglossaryentry{Org-mode}{name=Org-mode,
    description={Org-mode is an organizing mode for the editor GNU Emacs}}

\newglossaryentry{Grid5000}{name=Grid5000,
    description={Grid5000 is a large scale platform for experiment in computer
    science}}

\newglossaryentry{R}{name=R,
    description={R is a programming language for statistical computing}}

\newglossaryentry{R-markdown}{name=R-markdown,
    description={R-markdown is a combination of the R and markdown languages
    that allow to mix statistical analysis with formated comments}}
