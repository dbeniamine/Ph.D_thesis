%!TEX encoding=UTF-8 Unicode
%!TEX encoding=UTF-8 Unicode
%tweeks on pdf version so everybody is happy
%\pdfminorversion=4 % for facile.cines.fr
%\pdfcompresslevel=0 % Not needed

\documentclass[xcolor={usenames,dvipsnames},hyperref={pdfusetitle}]{beamer}
\usepackage{lmodern}

%=========================Language and encoding ==============================

\usepackage[utf8]{inputenc}
\usepackage[english]{babel}
\usepackage[T1]{fontenc}
% Fix size errors due to T1 in bbl file
\usepackage{fix-cm}
\usepackage{siunitx}
%=============================================================================

%========================= Todo notes  =======================================

%!TEX encoding=UTF-8 Unicode
\usepackage{xspace}

% Todo notes
\newcommand{\DB}[1]{\todo[author=David,inline]{#1}}
\newcommand{\DBm}[1]{\todo[author=David]{#1}}
\newcommand{\GH}[1]{\todo[author=Guillaume,inline]{#1}}
\newcommand{\GHm}[1]{\todo[author=Guillaume]{#1}}

% Usefull stuff
\newcommand{\Input}[1]{\input{tex/#1}}

% References
\newcommand{\fig}[1]{Figure~\ref{fig:#1}}
\newcommand{\tbl}[1]{Table~\ref{tab:#1}}
\newcommand{\alg}[1]{Algorithm~\ref{alg:#1}}
\newcommand{\lstr}[1]{Listing~\ref{lst:#1}}
\newcommand{\sect}[1]{Section~\ref{sec:#1}}
\newcommand{\chap}[1]{Chapter~\ref{chap:#1}}
% Bench
\newcommand{\FT}{\texttt{FT}\xspace}
\newcommand{\BT}{\texttt{BT}\xspace}
\newcommand{\IS}{\texttt{IS}\xspace}
\newcommand{\DC}{\texttt{DC}\xspace}
\newcommand{\MG}{\texttt{MG}\xspace}
\newcommand{\LU}{\texttt{LU}\xspace}
\newcommand{\UA}{\texttt{UA}\xspace}
\newcommand{\EP}{\texttt{EP}\xspace}
\newcommand{\SP}{\texttt{SP}\xspace}
\newcommand{\CG}{\texttt{CG}\xspace}
\newcommand{\Ondes}{\texttt{Ondes3D}\xspace}


\usepackage{todonotes}
\presetkeys{todonotes}{inline}{}

%=============================================================================

%========================= Figures ===========================================

\usepackage[]{caption}
\usepackage[]{subcaption}
\usepackage{graphicx} % support the \includegraphics command and options
\graphicspath{{./img/}{./style/}{./tikz/}}
%!TEX encoding=UTF-8 Unicode

% Libraries
\usepackage{tikz}
\usetikzlibrary{shapes,arrows,decorations,decorations.pathreplacing,decorations.markings,fit}
\usetikzlibrary{positioning,backgrounds,calc,patterns}

\usepackage{pgfplots}
\usepgfplotslibrary{fillbetween}

\pgfplotsset{
     compat=newest,
     samples=100
}

%!TEX encoding=UTF-8 Unicode
\usepackage{algorithm}
\usepackage{algpseudocode}
\usepackage{varwidth} % for algorithm in tikz node

\algblockdefx[]{Function}{EndFunction}
[2]{\algorithmicfunction\ \textproc{#1}{(#2)}}%
{\algorithmicend\ \algorithmicfunction}%
\algnewcommand\Callp[2]{\textproc{#1}(#2)}%


\usepackage{epstopdf}
\usepackage{booktabs}
\usepackage{multirow}
%\usepackage{subcaption}

%=============================================================================

%=============================================================================

%========================= Hyperref ==========================================


%\hypersetup{
%    colorlinks=false, %colore les liens
%    breaklinks=true, %permet le retour à la ligne dans les liens trop longs
%    urlcolor= blue, %couleur des hyperliens
%    %linkcolor= black, %couleur des liens internes
%    bookmarksopen=true,
%    citecolor=black,
%}
%=============================================================================

%========================= Other useful includes =============================

\usepackage{ifthen}
\usepackage[absolute,overlay]{textpos} %to set some blocks position
%=============================================================================

%========================= Beamer theme =====================================

%Stuff for printable version
\mode<handout>{
    \usetheme{default}
    \setbeamercolor{background canvas}{bg=black!5}
    \pgfpagesuselayout{4 on 1}[a4paper,landscape,border shrink=2.5mm]
}

\usetheme{AntibesCompact}

\definecolor{INstruct}{HTML}{82A382}
\setbeamercolor{structure}{fg=INstruct}

\newcommand{\alertitem}{\item<+-|alert@+>}
\newcommand\alertblockat[3]{
    \alt#1{
        \begin{alertblock}{#2}
            #3
        \end{alertblock}
    }{
        \begin{block}{#2}
            #3
        \end{block}
    }
}

%=============================================================================

%========================= Title frame  ======================================
\title[]{Analyzing the memory behavior of parallel scientific applications}
\author[David Beniamine]{\textbf{David Beniamine}}
\institute[Polaris / Datamove]{
    \includegraphics[height=.10\textheight]{img/logoUGA.jpg}
    \quad
    \includegraphics[width=.10\textwidth]{img/LIG_coul.jpg}
    \quad
    \includegraphics[width=.15\textwidth]{img/inria.jpg}
    \quad
    \includegraphics[width=.18\textwidth]{img/polaris.png}
    \quad
    \includegraphics[width=.18\textwidth]{img/datamove.png}
}


\newcommand{\enableTocAtSection}{
    \AtBeginSection[]
    {
        \ifthenelse{\boolean{sectiontoc}}{
            \begin{frame}<beamer>
                \frametitle{Outline}
                \tableofcontents[currentsection,currentsubsection]
            \end{frame}
        }
    }
    \AtBeginSubsection[]
    {
        \ifthenelse{\boolean{sectiontoc}}{
            \begin{frame}<beamer>
                \frametitle{Outline}
                \tableofcontents[currentsection,currentsubsection]
            \end{frame}
        }
    }
}

%=============================================================================

\begin{document}
%========================= Title and outlines ================================
\begin{frame}{}
    \titlepage
\small
{\centering\itshape Jury members\par}
\begin{tabular}[t]{@{}l@{\hspace{3pt}}p{.45\textwidth}@{}}
President: & Pr, Martin Quinson\\
Reviewers: & Pr, Jes\'us Labarta Mancho \\
& Pr,  Raymond Namyst \\
Examiners: & Dr, Lucas M. Schnorr \\
\end{tabular}%
\begin{tabular}[t]{@{}l@{\hspace{3pt}}p{.45\textwidth}@{}}
Supervisors: & Pr, Bruno Raffin \\
 & Dr, Guillaume Huard
\end{tabular}%
\end{frame}

\newboolean{sectiontoc}
\setboolean{sectiontoc}{true} % default to true

%=============================================================================

%========================= Real presentation =================================

\section{Context}

\begin{frame}{Scientific applications and performances}
    \todo{Put something here}
\end{frame}

\begin{frame}{Improving sequential computer performances}
    \begin{columns}
        \begin{column}{.45\textwidth}
            \centering
            %!TEX encoding=UTF-8 Unicode
%Palette GnBu 5 col + white
\definecolor{ColPU}{HTML}{FFFFFF}
\definecolor{ColCore}{HTML}{F0F9E8}
\definecolor{ColL1}{HTML}{BAE4BC}
\definecolor{ColL2}{HTML}{7BCCC4}
\definecolor{ColL3}{HTML}{43A2CA}
\definecolor{ColM}{HTML}{0868AC}
\definecolor{ColS}{HTML}{FFFFFF}

\pgfdeclarelayer{bg}
\pgfdeclarelayer{bbg}
\pgfdeclarelayer{bbbg}
\pgfsetlayers{bbbg,bbg,bg,main}


\tikzset{
    box/.style={
        shape=rectangle,
        text centered,
        draw,
    },
}

\begin{tikzpicture}[font=\small, every pic/.style={scale=.9}]
    \node[box,fill=ColPU] (PU-0) at (0,0) {Thread};
    \node[minimum width=3.3em] (name) at ($(PU-0)+(0,1)$) {Core};

    \begin{pgfonlayer}{bg}
        \node[box,fill=ColCore, fit=(name) (PU-0) ] (core-0) {};
    \end{pgfonlayer}

    \node (cache) at ($(core-0.north)+(0,1)$) {};

    \uncover<3->{
        \draw (core-0.north) -- (cache);

        \draw[fill=ColL3] ($(core-0.north west)+(0,.5)$) rectangle
            ($(core-0.north east)+(0,1)$) node[pos=.5] {Cache};
    }

    \node (cpu-name) at ($(cache)+(0,.5)$) {CPU};

    \begin{pgfonlayer}{bbg}
        \node[box,fill=ColS,fit=(core-0) (cpu-name)] (cpu) {};
    \end{pgfonlayer}

    \draw[line width=.5em] (cpu.north) -- ($(cpu.north)+(0,1)$);

    \draw[fill=ColM, text=white] ($(cpu.north west)+(-.5,1)$) rectangle
        ($(cpu.north east)+(.5,2)$) node[pos=.5]{Memory};


\end{tikzpicture}
% vim: et si sta lbr  sw=4 ts=4 spelllang=en_us

        \end{column}
        \begin{column}{.45\textwidth}
            \pause
            \begin{block}{Increase cpu frequency}
                \begin{itemize}[<+->]
                    \item Done for years
                    \alertitem Gap CPU / Memory
                    \begin{itemize}
                        \alertitem Add memory caches
                    \end{itemize}
                    \alertitem Physical limits
                        \begin{itemize}
                            \alertitem Energy consumption
                            \alertitem Heat dissipation
                        \end{itemize}
                \end{itemize}
            \end{block}
            \uncover<8->{
                \begin{alertblock}{Build parallel machines}
                    Writing efficient code becomes complex
                \end{alertblock}
            }
        \end{column}
    \end{columns}
\end{frame}

\begin{frame}{An simple example}
    \centering
    \scalebox{.6}{
        %!TEX encoding=UTF-8 Unicode
%Palette PuOr 4 cols
\definecolor{ColI}{HTML}{E66101}
\definecolor{ColJ}{HTML}{FDB863}
\definecolor{ColK}{HTML}{B2ABD2}

\pgfdeclarelayer{background}
\pgfdeclarelayer{foreground}
\pgfsetlayers{background,foreground}


\tikzstyle{PrimaryA}   = [-latex,very thick]
\tikzstyle{SecondaryA} = [-latex,very thick,dashed]
\tikzstyle{SwapA} = [latex-latex, thick,dotted]

\newcommand{\coli}[1]{\textcolor{ColI}{#1}}
\newcommand{\colj}[1]{\textcolor{ColJ}{#1}}
\newcommand{\colk}[1]{\textcolor{ColK}{#1}}

\tikzset{
    algorithm/.style={
        shape=rectangle,
        alias=this,
        append after command = {
            \pgfextra{
              % Top and bottom lines
                \draw [] ($(this.north west)+(0,.5)$) -- ($(this.north east)+(0,.5)$);
                \node [anchor=west] at ($(this.north west)+(0,0.25)$) {\textbf{Algorithm} #1};
                \draw [] (this.north west) -- (this.north east);
                \draw [] (this.south west) -- (this.south east);
            }
        }
    },
    matgrid/.style args={#1#2}{
        %#1: name #2: size
        alias=this,
        append after command ={
            \pgfextra{
                \draw (this) grid ($(this)+(#2,#2)$);
                % Four corners
                \coordinate (m#1-00) at ($(this)+(0.5,0.5)$);
                \coordinate (m#1-0N) at ($(this)+(0.5,#2-0.5)$);
                \coordinate (m#1-N0) at ($(this)+(#2-0.5,0.5)$);
                \coordinate (m#1-NN) at ($(this)+(#2-0.5,#2-0.5)$);

                \node (#1) at ($(m#1-00)+(-1,0)$){\textbf{#1}};

                \node at ($(m#1-0N)+(-.2,.2)$)  {0};
                \node at ($(m#1-00)+(0,-.2)$) {N-1};
                \node at ($(m#1-NN)+(.2,0)$)  {N-1};

            }
        }
    }
}



\begin{tikzpicture}[font=\small]

    \begin{pgfonlayer}{background}
        \node[algorithm=Matrix multiplication] at (2.5,8.5){%
            \begin{varwidth}{\linewidth}
                \begin{algorithmic}
                    \For{\coli{i in 0..N-1}}
                    \For{\colj{j in 0..N-1}}
                    \For{\colk{k in 0..N-1}}
                    \State C[\coli{i},\colj{j}] += A[\coli{i},\colk{k}] * B[\colk{k},\colj{j}]
                            \EndFor
                        \EndFor
                    \EndFor
                \end{algorithmic}%
            \end{varwidth}%
        };

        \coordinate (cj) at (.6,9.3);
        \coordinate (cjint) at (.1,9.3);
        \coordinate (ckint) at (.1,8.9);
        \coordinate (ck) at (.6,8.9);
        \uncover<2->{
            \path[draw,SwapA] (cj) .. controls (cjint) and (ckint) ..(ck);
        }

        \node[matgrid={A}{5}] at (0,0){};
        \node[matgrid={B}{5}] at (6,6){};
        \node[matgrid={C}{5}] at (6,0){};

    \end{pgfonlayer}

    % Indexes

    \begin{pgfonlayer}{foreground}
        %% A
        \draw[PrimaryA,ColK]   (mA-0N) -- node [above] {\textbf{k}} (mA-NN);
        \draw[SecondaryA,ColI] (mA-0N) -- node [left]  {\textbf{i}} (mA-00);

        %% B
        \draw[PrimaryA,ColK]   (mB-0N) -- node(bk) [left]  {\textbf{k}} (mB-00);
        \draw[SecondaryA,ColJ] (mB-0N) -- node(bj) [above] {\textbf{j}} (mB-NN);

        \uncover<2->{
            \draw[SwapA] (bj.south) edge[out=-90,in=0] (bk.east);
        }

        %% C
        \draw[PrimaryA,ColJ]   (mC-0N) -- node [above] {\textbf{j}} (mC-NN);
        \draw[SecondaryA,ColI] (mC-0N) -- node [left]  {\textbf{i}} (mC-00);
    \end{pgfonlayer}

\end{tikzpicture}
% vim: et si sta lbr  sw=4 ts=4 spelllang=en_us

    }
\end{frame}

\begin{frame}{Parallel and NUMA machines}
    \centering
    \scalebox{.6}{
        %!TEX encoding=UTF-8 Unicode

\tikzset{
    box/.style={
        shape=rectangle,
        draw,
    },
    pics/core/.style args={#1#2}{
        % Args: #1: nb PU, #2 core id
        code={
            \pgfmathtruncatemacro{\pmin}{#1*#2}
            \pgfmathtruncatemacro{\pmax}{\pmin+#1-1}
            %PUs
            \foreach \pu in {\pmin,...,\pmax}{
                \pgfmathtruncatemacro{\pstep}{\pu-\pmin}
                \node[box] (PU-\pu)at ($(0,0)+(0,-.7*\pstep)$) {PU\#\pu};
            }
            %Core ID
            \node (name) at ($(PU-\pmin)!0.5!(PU-\pmax)+(0,1)$) {Core\##2};
            % L1
            \node[box,inner sep=2pt, fit=(name) (PU-\pmin) (PU-\pmax)] (box) {};
            \draw ($(box.north west)+(0,.1)$) rectangle ($(box.north east)+(0,.5)$)%
                node[pos=.5] (l1) {L1};
            % Coordinate for drawing L2
            \coordinate (core-#2-w) at ($(box.north west)+(0,.5)$);
            \coordinate (core-#2-e) at ($(box.north east)+(0,.5)$);
            % links
            \draw (box.north) -- (l1.south);
        }
    },
    pics/l2group/.style args={#1#2}{
        % Args: #1: nb Cores, #2 group id
        code={
            \pgfmathtruncatemacro{\cmin}{#1*#2}
            \pgfmathtruncatemacro{\cmax}{\cmin+#1-1}
            % Cores
            \foreach \core in {\cmin,...,\cmax}{
                \pgfmathtruncatemacro{\cstep}{\core-\cmin}
                \draw ($(0,0)+(1.6*\cstep,0)$) pic {core={2}{\core}};
            }
            % L2
            \draw ($(core-\cmin-w)+(0,.1)$) rectangle ($(core-\cmax-e)+(0,.5)$)%
                node[pos=.5]{L2};
            % Coordinates for L3
            \coordinate (l2g-#2-w) at ($(core-\cmin-w)+(0,.5)$);
            \coordinate (l2g-#2-e) at ($(core-\cmax-e)+(0,.5)$);
            %links
            \foreach \core in {\cmin,...,\cmax}{
                \draw ($(core-\core-w)!.5!(core-\core-e)$) --
                    ($(core-\core-w)!.5!(core-\core-e)+(0,.1)$);
            }
        }
    },
    pics/socket/.style args={#1#2}{
        % Args: #1: nb l2 groups, #2 socket id
        code={
            \pgfmathtruncatemacro{\lmin}{#1*#2}
            \pgfmathtruncatemacro{\lmax}{\lmin+#1-1}
            % Cores
            \foreach \lg in {\lmin,...,\lmax}{
                \draw ($(0,0)+(3.2*\lg,0)$) pic {l2group={2}{\lg}};
            }
            % L2
            \draw ($(l2g-\lmin-w)+(0,.1)$) rectangle ($(l2g-\lmax-e)+(0,.5)$)%
                node[pos=.5]{L3};
            % Coordinates for Mem
            \coordinate (s-#2-w) at ($(l2g-\lmin-w)+(0,.5)$);
            \coordinate (s-#2-e) at ($(l2g-\lmax-e)+(0,.5)$);
            % links
            \foreach \lg in {\lmin,...,\lmax}{
                \draw ($(l2g-\lg-w)!.5!(l2g-\lg-e)$) --
                    ($(l2g-\lg-w)!.5!(l2g-\lg-e)+(0,.1)$);
            }
            % CPU
            \node (minnode) at ($(-.7,-1)+(3.2*\lmin,0)$) {};
            \node (maxnode) at ($(l2g-\lmax-e)+(-.1,1)$) {};
            \node [anchor=west] (sockname) at ($(l2g-\lmin-w)+(0,.8)$) {Socket \##2};

            \node[box,fit=(minnode) (maxnode)] (cpu-#2) {};
            % Memory
            \draw ($(s-#2-w)+(0,1.5)$) rectangle ($(s-#2-e)+(0,2.5)$)%
                node[pos=.5]{Memory \##2};

            \draw ($(s-#2-w)!.5!(s-#2-e)+(0,.75)$) -- ($(s-#2-w)!.5!(s-#2-e)+(0,1.5)$);

        }
    },
}

\begin{tikzpicture}[font=\small, every pic/.style={scale=1}]
    \pic at (0,0) {socket={2}{0}};
    \pic at (.5,0) {socket={2}{1}};
    % Ugly:
    \draw (2.5,-1.25) |- (7,-2);
    \draw (7,-2) -| (9.5,-1.25);
\end{tikzpicture}
% vim: et si sta lbr  sw=4 ts=4 spelllang=en_us

    }
    \pause
\end{frame}

\begin{frame}{Research statement}
    \begin{alertblock}{Statement}
        How can we analyze the memory behavior of an application to optimize its performances ?
    \end{alertblock}
    \pause
    \begin{block}{Challenges}
        \begin{itemize}[<+->]
                \alertitem Collect memory traces
                \begin{itemize}
                    \item \textbf{Complete}: do not miss part of the address space
                    \item \textbf{Precise}:  enough to detect patterns
                    \item \textbf{Detailed}: embed all meta data about the access
                \end{itemize}
                \alertitem Visualize memory traces
                \alertitem Take advantage of the obtained knowledge
        \end{itemize}
    \end{block}
\end{frame}

\enableTocAtSection
\begin{frame}{Outline}
    \tableofcontents
\end{frame}

\section{State of the art}

\begin{frame}{Generic performance analysis tools}
    \alertblockat{<1>}{Low level trace collection}{
        \begin{itemize}
            \item Likwid~\cite{Treibig10LIKWID}
            \item PAPI~\cite{Browne00Portable}
        \end{itemize}
    }
    \pause
    \alertblockat{<2>}{Higher level tools}{
        \begin{itemize}
            \item VTune~\cite{Reinders05VTune}
            \item HPCToolkit~\cite{Adhianto10HPCTOOLKIT}
            \item PARAVER~\cite{Pillet95PARAVER}
            \item MAQAO~\cite{Djoudi05MAQAO}
        \end{itemize}
    }
    \pause
    \begin{alertblock}{Limitations}
        Indirect information on the memory usage using  CPU performance counters
    \end{alertblock}
\end{frame}

\begin{frame}{Existing memory profiling tools}
    \begin{block}{Instruction sampling~\cite{Drongowski07Instructionbased,Levinthal09Performance}}
        \begin{itemize}
            \item  MemPhis~\cite{McCurdy10Memphis}
            \item  MemProf~\cite{Lachaize12MemProf}
            \item  HPCToolkit extension~\cite{Liu14Tool}
            \item  MemAxes / Mitos~\cite{Gimenez14Dissecting}
        \end{itemize}
    \end{block}
    \pause 
    \begin{alertblock}{Limitations}
        \begin{itemize}
            \item Not complete: miss significant part of memory
            \item Not precise enough for pattern analysis
        \end{itemize}
    \end{alertblock}
\end{frame}

\section{Analyzing the global memory behavior}

\subsection{Tabarnac}

\begin{frame}{Trace collection mechanism}
    \begin{itemize}[<+-|alert@+>]
        \item Aims at improving performances on NUMA machines
        \item Pin~\cite{Luk05Pin} instrumentation
        \item Based on Numalyze~\cite{Diener15Characterizing}
        \item Collects 1 counter per page and per threads
        \item Lock free
        \item Differentiate access types (reads / writes)
        \item Retrieve data structure information
    \end{itemize}
\end{frame}

\begin{frame}{Visualizations}
    \alt<1>{
        \begin{columns}
            \begin{column}{.45\linewidth}
                \centering
                \includegraphics[width=\linewidth]{img/tabarnac/example_sz.png}
            \end{column}
            \begin{column}{.45\linewidth}
                \includegraphics[width=\linewidth]{img/tabarnac/example_rw.png}
            \end{column}
        \end{columns}
    }{
        \centering
        \alt<2>{
            \includegraphics[height=.7\textheight]{img/tabarnac/example_ft.png}
            \begin{block}{}
                First touch distribution for one data structure
            \end{block}
        }{
            \includegraphics[height=.75\textheight]{img/tabarnac/is_b_kb2_orig.png}
            \begin{block}{}
                Accesses distribution for one data structure
            \end{block}
        }
    }
    \pause
    \pause
\end{frame}

\subsection{Using Tabarnac to optimize a well known benchmark}

\begin{frame}{First analysis}
\begin{figure}[htb]
    \begin{columns}
        \begin{column}{.45\linewidth}
            \includegraphics[width=\linewidth]  {tabarnac/is_b_kb2_orig}
        \end{column}
        \begin{column}{.45\linewidth}
            \includegraphics[width=\linewidth]  {tabarnac/is_b_kb1_orig}
        \end{column}
    \end{columns}
\end{figure}

\end{frame}

\begin{frame}{Issue and optimization}
    \alt<1-2>{
        \scalebox{.8}{
            \centering
            \pgfmathdeclarefunction{gauss}{1}{%
    \pgfmathparse{1/(1.4*sqrt(2*pi))*exp(-((#1-4)^2)/(2*1.4^2))}%
}

%\pgfdeclarelayer{back}
%\pgfdeclarelayer{bb}
%\pgfsetlayers{bb,back,main}

%Palette 4-class paired
\definecolor{Col0}{HTML}{A6CEE3}
\definecolor{Col1}{HTML}{1F78B4}
\definecolor{Col2}{HTML}{B2DF8A}
\definecolor{Col3}{HTML}{33A02C}

\begin{tikzpicture}
\begin{axis}[
        axis x line=bottom,
        axis y line=left,
        xtick=\empty,
        ytick=\empty,
        ylabel={Intensity of accesses},
        xlabel={Page number},
        ymin=-.01,
        xmin=-.1,
        xmax=8.1,
        legend style={at={(1.1,1)}, anchor= north east}
     ]
    \addplot[name path=f,domain={0:8},forget plot] {gauss(x)};
    \addplot[name path=x,domain={0:8},forget plot] {0};
    %\pgfplotsinvokeforeach{0,...,3}{
    %%            \addlegendentry{Thread #1}
    %    \addplot[color= Col#1,forget plot] fill between[of=f and x,
    %        soft clip={(0,0) rectangle (8,#1*.25)},];
    %}

\end{axis}
\end{tikzpicture}

        }
        \pause
        \begin{block}{Original code}
            \#pragma omp for schedule(dynamic)
        \end{block}
    }{
        \scalebox{.8}{
            \centering
            \pgfmathdeclarefunction{gauss}{1}{%
    \pgfmathparse{1/(1.4*sqrt(2*pi))*exp(-((#1-4)^2)/(2*1.4^2))}%
}

\definecolor{Col0}{HTML}{A6CEE3}
\definecolor{Col1}{HTML}{1F78B4}
\definecolor{Col2}{HTML}{B2DF8A}
\definecolor{Col3}{HTML}{33A02C}

\begin{tikzpicture}
\begin{axis}[
        axis x line=bottom,
        axis y line=left,
        xtick=\empty,
        ytick=\empty,
        ylabel={Intensity of accesses},
        xlabel={Page number},
        ymin=-.01,
        xmin=-.1,
        xmax=8.1,
        legend style={at={(1.1,1)}, anchor= north east}
     ]
    \addplot[name path=f,domain={0:8},forget plot] {gauss(x)};
    \addplot[name path=x,domain={0:8},forget plot] {0};
    \pgfplotsinvokeforeach{0,...,3}{
        \addplot[dashed,forget plot] coordinates {(#1+4,0) (#1+4,{gauss(#1+4)})};
        \addplot[dashed,forget plot] coordinates {(#1,0) (#1,{gauss(#1)})};

        \addplot[color= Col#1,forget plot] fill between[of=f and x,
            soft clip={(#1,0) rectangle (#1+1,1) },];
            \addplot[color= Col#1] fill between[of=f and x,
            soft clip={(#1+4,0) rectangle (#1+5,1)},];
        \addlegendentry{Thread #1}
    }

    \addplot[solid,very thick,white] coordinates {(4,0) (4,{gauss(4)})};
\end{axis}
\end{tikzpicture}

        }        \begin{block}{Modified code}
            \#pragma omp for schedule(static, size/(2*num\_threads))
        \end{block}
    }
    \pause
\end{frame}

\begin{frame}{Second analysis}
    \begin{columns}
        \begin{column}{.45\linewidth}
            \includegraphics[width=\linewidth]  {tabarnac/is_b_kb2_modif}
        \end{column}
        \begin{column}{.45\linewidth}
            \includegraphics[width=\linewidth]  {tabarnac/is_b_kb1_modif}
        \end{column}
    \end{columns}
\end{frame}

\begin{frame}{Results}
    \includegraphics[width=\linewidth]{tabarnac/is_exectime}
\end{frame}

\begin{frame}{Conclusions}
    \begin{itemize}[<+-|alert@+>]
        \item Collaboration with M. Diener, UFRGS, P.O.A, Brasil
        \item Improved trace collection tool
        \item Simple yet meaningful visualization
        \item Enables significant optimizations
        \item Published at VPA'15~\cite{Beniamine15TABARNAC}
    \end{itemize}
    \uncover<6->{
        \begin{alertblock}{Limitations}
            \begin{itemize}
                \item No temporal informations
                \item Cannot detect precise sharing
            \end{itemize}
        \end{alertblock}
    }
\end{frame}

\section{Collecting and analyzing fine grained traces}

\subsection{Moca an efficient memory trace collection system}

\begin{frame}{Background knowledge}
    <++>
\end{frame}

\begin{frame}{Design}
    <++>
\end{frame}

\begin{frame}{Handling page faults}
    <++>
\end{frame}


\begin{frame}{Experimental setup}
    <++>
\end{frame}

\begin{frame}{Experimental results: trace precision}
    <++>
\end{frame}

\begin{frame}{Experimental results: overhead}
    <++>
\end{frame}

\subsection{Visualizing and analyzing Moca traces}

\begin{frame}{Framesoc / Ocelotl}
    <++>
\end{frame}

\begin{frame}{Example: Matrix multiplication}
    <++>
\end{frame}

\begin{frame}{Limits}
    <++>
\end{frame}

\begin{frame}{Programmatic approach}
    <++>
\end{frame}

\begin{frame}{Example: ??}
    <++>
\end{frame}

\section{Conclusions and perspectives}

\begin{frame}{Global overview}
    \cite{Beniamine15TABARNAC}
    <++>
\end{frame}

\begin{frame}{Fine trace collecion}
    \cite{Beniamine16Moca}
    <++>
\end{frame}

\begin{frame}{Trace analysis}
    <++>
\end{frame}

\begin{frame}{Perspectives}
    <++>
\end{frame}

\newcounter{finalframe}
\setcounter{finalframe}{\value{framenumber}}
%Last numbered frame go here

\begin{frame}{Perspectives}
\end{frame}
%=============================================================================

%=============================================================================
%Uncomment next lines for uncounted backup slides & biblio
\section*{Bibliography}
%
\bibliographystyle{apalike}
\bibliography{biblio}

%========================= Backup slides =====================================
\section*{Hidden slides}
%put this line before each frame
%\setcounter{framenumber}{\value{finalframe}}

\subsection*{Tabarnac}

\setcounter{framenumber}{\value{finalframe}}
\begin{frame}{Experimental setup}
    \small
    \centering
    \begin{tabular}{lccccc}
        \toprule
        & \multicolumn{5}{c}{\textbf{Hardware totals}}\\
        \cmidrule(lr){2-6}
        & Nodes & Threads & Vendor & Model & Memory \\
        \cmidrule(lr){2-6}
        \texttt{Turing}   & $4$ & $64$ & Intel & Xeon X7550   & \SI{128}{Gib} \\
        \texttt{Idfreeze} & $8$ & $48$ & AMD   & Opteron 6174 & \SI{256}{Gib}\\
        \midrule
        & \multicolumn{5}{c}{\textbf{Hardware per node}}\\
        \cmidrule(lr){2-6}
        & Cores & Threads & Frequency & L3 Cache & Memory \\
        \cmidrule(lr){2-6}
        \texttt{Turing}   & $8$ & $16$ & \SI{2.00}{Ghz}& \SI{18}{Mib} & \SI{32}{Gib} \\
        \texttt{Idfreeze} & $6$ & $6$  & \SI{2.20}{Ghz}& \SI{12}{Mib} & \SI{32}{Gib}\\
        \midrule
        & \multicolumn{5}{c}{\textbf{Software}}\\
        \cmidrule(lr){2-6}
        & Kernel & \multicolumn{2}{c}{Distribution} &
        \multicolumn{2}{c}{Bios configurations} \\
        \cmidrule(lr){2-6}
        \texttt{Turing}   & Linux 3.13 & \multicolumn{2}{c}{Ubuntu 12.04} &
        \multicolumn{2}{c}{Hyper threading} \\
        \texttt{Idfreeze} & Linux 3.2 & \multicolumn{2}{c}{Debian Jessie} &
        \multicolumn{2}{c}{No hyper threading}\\
        \bottomrule
    \end{tabular}
\end{frame}

\setcounter{framenumber}{\value{finalframe}}
\begin{frame}{Tabarnac overhead}
    \includegraphics[width=\linewidth]{tabarnac/tool-ovh.pdf}
\end{frame}

%=============================================================================
\end{document}




