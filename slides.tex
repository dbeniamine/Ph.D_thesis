%!TEX encoding=UTF-8 Unicode
%!TEX encoding=UTF-8 Unicode
%tweeks on pdf version so everybody is happy
%\pdfminorversion=4 % for facile.cines.fr
%\pdfcompresslevel=0 % Not needed

\documentclass[xcolor={usenames,dvipsnames},hyperref={pdfusetitle}]{beamer}
\usepackage{lmodern}
\beamertemplatenavigationsymbolsempty

%=========================Language and encoding ==============================

\usepackage[utf8]{inputenc}
\usepackage[english]{babel}
\usepackage[T1]{fontenc}
% Fix size errors due to T1 in bbl file
\usepackage{fix-cm}
\usepackage{siunitx}
%=============================================================================

%========================= Todo notes  =======================================

%!TEX encoding=UTF-8 Unicode
\usepackage{xspace}

% Todo notes
\newcommand{\DB}[1]{\todo[author=David,inline]{#1}}
\newcommand{\DBm}[1]{\todo[author=David]{#1}}
\newcommand{\GH}[1]{\todo[author=Guillaume,inline]{#1}}
\newcommand{\GHm}[1]{\todo[author=Guillaume]{#1}}

% Usefull stuff
\newcommand{\Input}[1]{\input{tex/#1}}

% References
\newcommand{\fig}[1]{Figure~\ref{fig:#1}}
\newcommand{\tbl}[1]{Table~\ref{tab:#1}}
\newcommand{\alg}[1]{Algorithm~\ref{alg:#1}}
\newcommand{\lstr}[1]{Listing~\ref{lst:#1}}
\newcommand{\sect}[1]{Section~\ref{sec:#1}}
\newcommand{\chap}[1]{Chapter~\ref{chap:#1}}
% Bench
\newcommand{\FT}{\texttt{FT}\xspace}
\newcommand{\BT}{\texttt{BT}\xspace}
\newcommand{\IS}{\texttt{IS}\xspace}
\newcommand{\DC}{\texttt{DC}\xspace}
\newcommand{\MG}{\texttt{MG}\xspace}
\newcommand{\LU}{\texttt{LU}\xspace}
\newcommand{\UA}{\texttt{UA}\xspace}
\newcommand{\EP}{\texttt{EP}\xspace}
\newcommand{\SP}{\texttt{SP}\xspace}
\newcommand{\CG}{\texttt{CG}\xspace}
\newcommand{\Ondes}{\texttt{Ondes3D}\xspace}


\usepackage{todonotes}
\presetkeys{todonotes}{inline}{}

%=============================================================================

%========================= Figures ===========================================

\usepackage[]{caption}
\usepackage[]{subcaption}
\usepackage{graphicx} % support the \includegraphics command and options
\graphicspath{{./img/}{./style/}{./tikz/}}
%!TEX encoding=UTF-8 Unicode

% Libraries
\usepackage{tikz}
\usetikzlibrary{shapes,arrows,decorations,decorations.pathreplacing,decorations.markings,fit}
\usetikzlibrary{positioning,backgrounds,calc,patterns}

\usepackage{pgfplots}
\usepgfplotslibrary{fillbetween}

\pgfplotsset{
     compat=newest,
     samples=100
}

%!TEX encoding=UTF-8 Unicode
\usepackage{algorithm}
\usepackage{algpseudocode}
\usepackage{varwidth} % for algorithm in tikz node

\algblockdefx[]{Function}{EndFunction}
[2]{\algorithmicfunction\ \textproc{#1}{(#2)}}%
{\algorithmicend\ \algorithmicfunction}%
\algnewcommand\Callp[2]{\textproc{#1}(#2)}%


\usepackage{epstopdf}
\usepackage{booktabs}
\usepackage{multirow}
%\usepackage{subcaption}

%=============================================================================

%=============================================================================

%========================= Hyperref ==========================================


%\hypersetup{
%    colorlinks=false, %colore les liens
%    breaklinks=true, %permet le retour à la ligne dans les liens trop longs
%    urlcolor= blue, %couleur des hyperliens
%    %linkcolor= black, %couleur des liens internes
%    bookmarksopen=true,
%    citecolor=black,
%}
%=============================================================================

%========================= Other useful includes =============================

\usepackage{ifthen}
\usepackage[absolute,overlay]{textpos} %to set some blocks position
%=============================================================================

%========================= Beamer theme =====================================

%Stuff for printable version
\mode<handout>{
    \usetheme{default}
    \setbeamercolor{background canvas}{bg=black!5}
    \pgfpagesuselayout{4 on 1}[a4paper,landscape,border shrink=2.5mm]
}

\usetheme{AntibesCompact}

\definecolor{INstruct}{HTML}{82A382}
\setbeamercolor{structure}{fg=INstruct}

\newcommand{\alertitem}{\item<+-|alert@+>}
\newcommand\alertblockat[3]{
    \alt#1{
        \begin{alertblock}{#2}
            #3
        \end{alertblock}
    }{
        \begin{block}{#2}
            #3
        \end{block}
    }
}

%=============================================================================

%========================= Title frame  ======================================
\title[]{Analyzing the memory behavior of parallel scientific applications}
\author[David Beniamine]{\textbf{David Beniamine}}
\institute[Polaris / Datamove]{
    \includegraphics[height=.10\textheight]{img/logoUGA.jpg}
    \quad
    \includegraphics[width=.10\textwidth]{img/LIG_coul.jpg}
    \quad
    \includegraphics[width=.15\textwidth]{img/inria.jpg}
    \quad
    \includegraphics[width=.18\textwidth]{img/polaris.png}
    \quad
    \includegraphics[width=.18\textwidth]{img/datamove.png}
}


\newcommand{\enableTocAtSection}{
    \AtBeginSection[]
    {
        \ifthenelse{\boolean{sectiontoc}}{
            \begin{frame}<beamer>
                \frametitle{Outline}
                \tableofcontents[currentsection,currentsubsection]
            \end{frame}
        }
    }
    \AtBeginSubsection[]
    {
        \ifthenelse{\boolean{sectiontoc}}{
            \begin{frame}<beamer>
                \frametitle{Outline}
                \tableofcontents[currentsection,currentsubsection]
            \end{frame}
        }
    }
}

%=============================================================================

\begin{document}
%========================= Title and outlines ================================
\begin{frame}{}
    \titlepage
\small
{\centering\itshape Jury members\par}
\begin{tabular}[t]{@{}l@{\hspace{3pt}}p{.45\textwidth}@{}}
President: & Pr, Martin Quinson\\
Reviewers: & Pr, Jes\'us Labarta Mancho \\
& Pr,  Raymond Namyst \\
Examiners: & Dr, Lucas M. Schnorr \\
\end{tabular}%
\begin{tabular}[t]{@{}l@{\hspace{3pt}}p{.45\textwidth}@{}}
Supervisors: & Pr, Bruno Raffin \\
 & Dr, Guillaume Huard
\end{tabular}%
\end{frame}

\newboolean{sectiontoc}
\setboolean{sectiontoc}{true} % default to true

%=============================================================================

%========================= Real presentation =================================

\section{Context}

\begin{frame}{Science and computers}
    %!TEX encoding=UTF-8 Unicode

% Layers

\definecolor{Computer}{HTML}{D95F02}
\definecolor{Science}{HTML}{1B9E77}
\definecolor{Theory}{HTML}{7570B3}

\tikzstyle{arr}  = [-latex,very thick]
\tikzset{
  filled box/.style = {
    shape = rectangle,
    draw  = #1,
    fill  = #1,
    text width=6em,
    centered,
    rounded corners},
}

%\newlength{\cornerlength}
%\setlength{\cornerlength}{.1}

\begin{tikzpicture}[font=\small,scale=1]
    \node[filled box=Computer] (computers)   at (0,3) {\only<5->{\textbf{More\newline performant\newline}} computers};
    \uncover<2->{
        \node[filled box=Science] (simu)        at (5,5) {\only<4->{\textbf{More complex\newline}} Simulation \newline and large scale \newline experiments};
        \path[arr,Computer] (computers.north) edge[out=90,in=180] (simu.west);
}
    \uncover<3->{
        \node[filled box=Theory] (th)          at (8,0)  {New theories};
        \path[arr,Science] (simu.east) edge[out=0,in=90] (th.north);
    }

    \uncover<4->{
        \path[arr,Theory] (th.west) edge[out=180,in=-75] (simu.south);
    }
 
    \uncover<5->{
        \path[arr,Science] (simu.south) edge[out=-105,in=0] (computers.east);
    }


\end{tikzpicture}
% vim: et si sta lbr  sw=4 ts=4 spelllang=en_us

\end{frame}

\begin{frame}{Improving sequential computer performances}
    \begin{columns}
        \begin{column}{.45\textwidth}
            \centering
            %!TEX encoding=UTF-8 Unicode
%Palette GnBu 5 col + white
\definecolor{ColPU}{HTML}{FFFFFF}
\definecolor{ColCore}{HTML}{F0F9E8}
\definecolor{ColL1}{HTML}{BAE4BC}
\definecolor{ColL2}{HTML}{7BCCC4}
\definecolor{ColL3}{HTML}{43A2CA}
\definecolor{ColM}{HTML}{0868AC}
\definecolor{ColS}{HTML}{FFFFFF}

\pgfdeclarelayer{bg}
\pgfdeclarelayer{bbg}
\pgfdeclarelayer{bbbg}
\pgfsetlayers{bbbg,bbg,bg,main}


\tikzset{
    box/.style={
        shape=rectangle,
        text centered,
        draw,
    },
}

\begin{tikzpicture}[font=\small, every pic/.style={scale=.9}]
    \node[box,fill=ColPU] (PU-0) at (0,0) {Thread};
    \node[minimum width=3.3em] (name) at ($(PU-0)+(0,1)$) {Core};

    \begin{pgfonlayer}{bg}
        \node[box,fill=ColCore, fit=(name) (PU-0) ] (core-0) {};
    \end{pgfonlayer}

    \node (cache) at ($(core-0.north)+(0,1)$) {};

    \uncover<3->{
        \draw (core-0.north) -- (cache);

        \draw[fill=ColL3] ($(core-0.north west)+(0,.5)$) rectangle
            ($(core-0.north east)+(0,1)$) node[pos=.5] {Cache};
    }

    \node (cpu-name) at ($(cache)+(0,.5)$) {CPU};

    \begin{pgfonlayer}{bbg}
        \node[box,fill=ColS,fit=(core-0) (cpu-name)] (cpu) {};
    \end{pgfonlayer}

    \draw[line width=.5em] (cpu.north) -- ($(cpu.north)+(0,1)$);

    \draw[fill=ColM, text=white] ($(cpu.north west)+(-.5,1)$) rectangle
        ($(cpu.north east)+(.5,2)$) node[pos=.5]{Memory};


\end{tikzpicture}
% vim: et si sta lbr  sw=4 ts=4 spelllang=en_us

        \end{column}
        \begin{column}{.45\textwidth}
            \pause
            \begin{block}{Increase cpu frequency}
                \begin{itemize}[<+->]
                    \item Done for years
                    \alertitem Gap CPU / Memory
                    \begin{itemize}
                        \alertitem Add memory caches
                    \end{itemize}
                    \alertitem Energy consumption
                \end{itemize}
            \end{block}
            \uncover<6->{
                \begin{alertblock}{Solution}
                    Build parallel machines
                \end{alertblock}
            }
        \end{column}
    \end{columns}
\end{frame}

\begin{frame}{An simple example}
    \centering
    \scalebox{.6}{
        %!TEX encoding=UTF-8 Unicode
%Palette PuOr 4 cols
\definecolor{ColI}{HTML}{E66101}
\definecolor{ColJ}{HTML}{FDB863}
\definecolor{ColK}{HTML}{B2ABD2}

\pgfdeclarelayer{background}
\pgfdeclarelayer{foreground}
\pgfsetlayers{background,foreground}


\tikzstyle{PrimaryA}   = [-latex,very thick]
\tikzstyle{SecondaryA} = [-latex,very thick,dashed]
\tikzstyle{SwapA} = [latex-latex, thick,dotted]

\newcommand{\coli}[1]{\textcolor{ColI}{#1}}
\newcommand{\colj}[1]{\textcolor{ColJ}{#1}}
\newcommand{\colk}[1]{\textcolor{ColK}{#1}}

\tikzset{
    algorithm/.style={
        shape=rectangle,
        alias=this,
        append after command = {
            \pgfextra{
              % Top and bottom lines
                \draw [] ($(this.north west)+(0,.5)$) -- ($(this.north east)+(0,.5)$);
                \node [anchor=west] at ($(this.north west)+(0,0.25)$) {\textbf{Algorithm} #1};
                \draw [] (this.north west) -- (this.north east);
                \draw [] (this.south west) -- (this.south east);
            }
        }
    },
    matgrid/.style args={#1#2}{
        %#1: name #2: size
        alias=this,
        append after command ={
            \pgfextra{
                \draw (this) grid ($(this)+(#2,#2)$);
                % Four corners
                \coordinate (m#1-00) at ($(this)+(0.5,0.5)$);
                \coordinate (m#1-0N) at ($(this)+(0.5,#2-0.5)$);
                \coordinate (m#1-N0) at ($(this)+(#2-0.5,0.5)$);
                \coordinate (m#1-NN) at ($(this)+(#2-0.5,#2-0.5)$);

                \node (#1) at ($(m#1-00)+(-1,0)$){\textbf{#1}};

                \node at ($(m#1-0N)+(-.2,.2)$)  {0};
                \node at ($(m#1-00)+(0,-.2)$) {N-1};
                \node at ($(m#1-NN)+(.2,0)$)  {N-1};

            }
        }
    }
}



\begin{tikzpicture}[font=\small]

    \begin{pgfonlayer}{background}
        \node[algorithm=Matrix multiplication] at (2.5,8.5){%
            \begin{varwidth}{\linewidth}
                \begin{algorithmic}
                    \For{\coli{i in 0..N-1}}
                    \For{\colj{j in 0..N-1}}
                    \For{\colk{k in 0..N-1}}
                    \State C[\coli{i},\colj{j}] += A[\coli{i},\colk{k}] * B[\colk{k},\colj{j}]
                            \EndFor
                        \EndFor
                    \EndFor
                \end{algorithmic}%
            \end{varwidth}%
        };

        \coordinate (cj) at (.6,9.3);
        \coordinate (cjint) at (.1,9.3);
        \coordinate (ckint) at (.1,8.9);
        \coordinate (ck) at (.6,8.9);
        \uncover<2->{
            \path[draw,SwapA] (cj) .. controls (cjint) and (ckint) ..(ck);
        }

        \node[matgrid={A}{5}] at (0,0){};
        \node[matgrid={B}{5}] at (6,6){};
        \node[matgrid={C}{5}] at (6,0){};

    \end{pgfonlayer}

    % Indexes

    \begin{pgfonlayer}{foreground}
        %% A
        \draw[PrimaryA,ColK]   (mA-0N) -- node [above] {\textbf{k}} (mA-NN);
        \draw[SecondaryA,ColI] (mA-0N) -- node [left]  {\textbf{i}} (mA-00);

        %% B
        \draw[PrimaryA,ColK]   (mB-0N) -- node(bk) [left]  {\textbf{k}} (mB-00);
        \draw[SecondaryA,ColJ] (mB-0N) -- node(bj) [above] {\textbf{j}} (mB-NN);

        \uncover<2->{
            \draw[SwapA] (bj.south) edge[out=-90,in=0] (bk.east);
        }

        %% C
        \draw[PrimaryA,ColJ]   (mC-0N) -- node [above] {\textbf{j}} (mC-NN);
        \draw[SecondaryA,ColI] (mC-0N) -- node [left]  {\textbf{i}} (mC-00);
    \end{pgfonlayer}

\end{tikzpicture}
% vim: et si sta lbr  sw=4 ts=4 spelllang=en_us

    }
\end{frame}

\begin{frame}{Parallel and NUMA machines}
    \centering
    \scalebox{.6}{
        %!TEX encoding=UTF-8 Unicode

\tikzset{
    box/.style={
        shape=rectangle,
        draw,
    },
    pics/core/.style args={#1#2}{
        % Args: #1: nb PU, #2 core id
        code={
            \pgfmathtruncatemacro{\pmin}{#1*#2}
            \pgfmathtruncatemacro{\pmax}{\pmin+#1-1}
            %PUs
            \foreach \pu in {\pmin,...,\pmax}{
                \pgfmathtruncatemacro{\pstep}{\pu-\pmin}
                \node[box] (PU-\pu)at ($(0,0)+(0,-.7*\pstep)$) {PU\#\pu};
            }
            %Core ID
            \node (name) at ($(PU-\pmin)!0.5!(PU-\pmax)+(0,1)$) {Core\##2};
            % L1
            \node[box,inner sep=2pt, fit=(name) (PU-\pmin) (PU-\pmax)] (box) {};
            \draw ($(box.north west)+(0,.1)$) rectangle ($(box.north east)+(0,.5)$)%
                node[pos=.5] (l1) {L1};
            % Coordinate for drawing L2
            \coordinate (core-#2-w) at ($(box.north west)+(0,.5)$);
            \coordinate (core-#2-e) at ($(box.north east)+(0,.5)$);
            % links
            \draw (box.north) -- (l1.south);
        }
    },
    pics/l2group/.style args={#1#2}{
        % Args: #1: nb Cores, #2 group id
        code={
            \pgfmathtruncatemacro{\cmin}{#1*#2}
            \pgfmathtruncatemacro{\cmax}{\cmin+#1-1}
            % Cores
            \foreach \core in {\cmin,...,\cmax}{
                \pgfmathtruncatemacro{\cstep}{\core-\cmin}
                \draw ($(0,0)+(1.6*\cstep,0)$) pic {core={2}{\core}};
            }
            % L2
            \draw ($(core-\cmin-w)+(0,.1)$) rectangle ($(core-\cmax-e)+(0,.5)$)%
                node[pos=.5]{L2};
            % Coordinates for L3
            \coordinate (l2g-#2-w) at ($(core-\cmin-w)+(0,.5)$);
            \coordinate (l2g-#2-e) at ($(core-\cmax-e)+(0,.5)$);
            %links
            \foreach \core in {\cmin,...,\cmax}{
                \draw ($(core-\core-w)!.5!(core-\core-e)$) --
                    ($(core-\core-w)!.5!(core-\core-e)+(0,.1)$);
            }
        }
    },
    pics/socket/.style args={#1#2}{
        % Args: #1: nb l2 groups, #2 socket id
        code={
            \pgfmathtruncatemacro{\lmin}{#1*#2}
            \pgfmathtruncatemacro{\lmax}{\lmin+#1-1}
            % Cores
            \foreach \lg in {\lmin,...,\lmax}{
                \draw ($(0,0)+(3.2*\lg,0)$) pic {l2group={2}{\lg}};
            }
            % L2
            \draw ($(l2g-\lmin-w)+(0,.1)$) rectangle ($(l2g-\lmax-e)+(0,.5)$)%
                node[pos=.5]{L3};
            % Coordinates for Mem
            \coordinate (s-#2-w) at ($(l2g-\lmin-w)+(0,.5)$);
            \coordinate (s-#2-e) at ($(l2g-\lmax-e)+(0,.5)$);
            % links
            \foreach \lg in {\lmin,...,\lmax}{
                \draw ($(l2g-\lg-w)!.5!(l2g-\lg-e)$) --
                    ($(l2g-\lg-w)!.5!(l2g-\lg-e)+(0,.1)$);
            }
            % CPU
            \node (minnode) at ($(-.7,-1)+(3.2*\lmin,0)$) {};
            \node (maxnode) at ($(l2g-\lmax-e)+(-.1,1)$) {};
            \node [anchor=west] (sockname) at ($(l2g-\lmin-w)+(0,.8)$) {Socket \##2};

            \node[box,fit=(minnode) (maxnode)] (cpu-#2) {};
            % Memory
            \draw ($(s-#2-w)+(0,1.5)$) rectangle ($(s-#2-e)+(0,2.5)$)%
                node[pos=.5]{Memory \##2};

            \draw ($(s-#2-w)!.5!(s-#2-e)+(0,.75)$) -- ($(s-#2-w)!.5!(s-#2-e)+(0,1.5)$);

        }
    },
}

\begin{tikzpicture}[font=\small, every pic/.style={scale=1}]
    \pic at (0,0) {socket={2}{0}};
    \pic at (.5,0) {socket={2}{1}};
    % Ugly:
    \draw (2.5,-1.25) |- (7,-2);
    \draw (7,-2) -| (9.5,-1.25);
\end{tikzpicture}
% vim: et si sta lbr  sw=4 ts=4 spelllang=en_us

    }
    \pause
\end{frame}

\begin{frame}{Research statement}
    \begin{alertblock}{Statement}
        How can we analyze the memory behavior of an application to optimize its performances ?
    \end{alertblock}
    \pause
    \begin{block}{Challenges}
        \begin{itemize}[<+->]
                \alertitem Collect memory traces
                \begin{itemize}
                    \item \textbf{Complete}: do not miss part of the address space
                    \item \textbf{Precise}:  enough to detect patterns
                    \item \textbf{Detailed}: embed all meta data about the access
                \end{itemize}
                \alertitem Visualize memory traces
                \alertitem Take advantage of the obtained knowledge
        \end{itemize}
    \end{block}
\end{frame}

\enableTocAtSection
\begin{frame}{Outline}
    \tableofcontents
\end{frame}

\section{State of the art}

\begin{frame}{Generic performance analysis tools}
    \alertblockat{<1>}{Low level trace collection libraries}{
        \begin{itemize}
            \item Likwid~\cite{Treibig10LIKWID}
            \item PAPI~\cite{Browne00Portable}
        \end{itemize}
    }
    \pause
    \alertblockat{<2>}{Higher level tools}{
        \begin{itemize}
            \item VTune~\cite{Reinders05VTune}
            \item HPCToolkit~\cite{Adhianto10HPCTOOLKIT}
            \item PARAVER~\cite{Pillet95PARAVER}
            \item MAQAO~\cite{Djoudi05MAQAO}
        \end{itemize}
    }
    \pause
    \begin{alertblock}{Limitations}
        Focus on CPU point of view
    \end{alertblock}
\end{frame}

\begin{frame}{Existing memory profiling tools}
    \begin{block}{Instruction sampling~\cite{Drongowski07Instructionbased,Levinthal09Performance}}
        \begin{itemize}
            \item  MemPhis~\cite{McCurdy10Memphis}
            \item  MemProf~\cite{Lachaize12MemProf}
            \item  HPCToolkit extension~\cite{Liu14Tool}
            \item  MemAxes / Mitos~\cite{Gimenez14Dissecting}
        \end{itemize}
    \end{block}
    \pause
    \begin{alertblock}{Limitations}
        \begin{itemize}
            \item Not complete: miss significant part of memory
            \item Not precise enough to detect patterns
        \end{itemize}
    \end{alertblock}
\end{frame}

\begin{frame}{Dynamic mapping tools~\cite{Corbet12Toward,Diener14kMAF}}
    \begin{block}{Principles}
        \begin{itemize}
            \item Collect informations on memory accesses online
            \item Move data close to thread using them
        \end{itemize}
    \end{block}
    \pause
    \begin{alertblock}{Limitations}
        \begin{itemize}
            \item Do not store the trace
            \item Do not help understanding issues
            \item Opportunities of optimization lost during learning process
            \item Cost of data move
        \end{itemize}
    \end{alertblock}
\end{frame}

\section{Analyzing the global memory behavior}

\subsection{Tabarnac}

\begin{frame}{Trace collection mechanism}
    \begin{block}{Numalyze~\cite{Diener15Characterizing}}
        \begin{itemize}
            \item Pin~\cite{Luk05Pin} instrumentation
            \item One counter per page and per threads
            \item Lock free
        \end{itemize}
    \end{block}
    \pause
    \begin{alertblock}{Tabarnac}
        \begin{itemize}
            \item Differentiate access types (reads / writes)
            \item Retrieve data structure information
                \begin{itemize}
                    \item Read binary for static data structures
                    \item Intercepts allocations
                    \item Uses debug flags to decide data structure names
                \end{itemize}
            \item Designed for improving performances on NUMA Machines
            \item Simple yet meaningful R visualizations
        \end{itemize}
    \end{alertblock}
\end{frame}

\begin{frame}{Visualizations}
    \alt<1>{
        \begin{columns}
            \begin{column}{.45\linewidth}
                \centering
                \includegraphics[width=\linewidth]{img/tabarnac/example_sz.png}
            \end{column}
            \begin{column}{.45\linewidth}
                \includegraphics[width=\linewidth]{img/tabarnac/example_rw.png}
            \end{column}
        \end{columns}
    }{
        \centering
        \alt<2>{
            \includegraphics[height=.7\textheight]{img/tabarnac/example_ft.png}
            \begin{block}{}
                First touch distribution for one data structure
            \end{block}
        }{
            \includegraphics[height=.75\textheight]{img/tabarnac/is_b_kb2_orig.png}
            \begin{block}{}
                Accesses distribution for one data structure
            \end{block}
        }
    }
    \pause
    \pause
\end{frame}

\subsection{Using Tabarnac to optimize a well known benchmark}

\begin{frame}{First analysis}
\begin{figure}[htb]
    \begin{columns}
        \begin{column}{.45\linewidth}
            \includegraphics[width=\linewidth]  {tabarnac/is_b_kb2_orig}
        \end{column}
        \begin{column}{.45\linewidth}
            \includegraphics[width=\linewidth]  {tabarnac/is_b_kb1_orig}
        \end{column}
    \end{columns}
\end{figure}

\end{frame}

\begin{frame}{Issue and optimization}
    \alt<1-2>{
        \centering
        \scalebox{.8}{
            \pgfmathdeclarefunction{gauss}{1}{%
    \pgfmathparse{1/(1.4*sqrt(2*pi))*exp(-((#1-4)^2)/(2*1.4^2))}%
}

%\pgfdeclarelayer{back}
%\pgfdeclarelayer{bb}
%\pgfsetlayers{bb,back,main}

%Palette 4-class paired
\definecolor{Col0}{HTML}{A6CEE3}
\definecolor{Col1}{HTML}{1F78B4}
\definecolor{Col2}{HTML}{B2DF8A}
\definecolor{Col3}{HTML}{33A02C}

\begin{tikzpicture}
\begin{axis}[
        axis x line=bottom,
        axis y line=left,
        xtick=\empty,
        ytick=\empty,
        ylabel={Intensity of accesses},
        xlabel={Page number},
        ymin=-.01,
        xmin=-.1,
        xmax=8.1,
        legend style={at={(1.1,1)}, anchor= north east}
     ]
    \addplot[name path=f,domain={0:8},forget plot] {gauss(x)};
    \addplot[name path=x,domain={0:8},forget plot] {0};
    %\pgfplotsinvokeforeach{0,...,3}{
    %%            \addlegendentry{Thread #1}
    %    \addplot[color= Col#1,forget plot] fill between[of=f and x,
    %        soft clip={(0,0) rectangle (8,#1*.25)},];
    %}

\end{axis}
\end{tikzpicture}

        }
        \pause
        \begin{block}{Original code}
            \#pragma omp for schedule(dynamic)
        \end{block}
    }{
        \centering
        \scalebox{.8}{
            \pgfmathdeclarefunction{gauss}{1}{%
    \pgfmathparse{1/(1.4*sqrt(2*pi))*exp(-((#1-4)^2)/(2*1.4^2))}%
}

\definecolor{Col0}{HTML}{A6CEE3}
\definecolor{Col1}{HTML}{1F78B4}
\definecolor{Col2}{HTML}{B2DF8A}
\definecolor{Col3}{HTML}{33A02C}

\begin{tikzpicture}
\begin{axis}[
        axis x line=bottom,
        axis y line=left,
        xtick=\empty,
        ytick=\empty,
        ylabel={Intensity of accesses},
        xlabel={Page number},
        ymin=-.01,
        xmin=-.1,
        xmax=8.1,
        legend style={at={(1.1,1)}, anchor= north east}
     ]
    \addplot[name path=f,domain={0:8},forget plot] {gauss(x)};
    \addplot[name path=x,domain={0:8},forget plot] {0};
    \pgfplotsinvokeforeach{0,...,3}{
        \addplot[dashed,forget plot] coordinates {(#1+4,0) (#1+4,{gauss(#1+4)})};
        \addplot[dashed,forget plot] coordinates {(#1,0) (#1,{gauss(#1)})};

        \addplot[color= Col#1,forget plot] fill between[of=f and x,
            soft clip={(#1,0) rectangle (#1+1,1) },];
            \addplot[color= Col#1] fill between[of=f and x,
            soft clip={(#1+4,0) rectangle (#1+5,1)},];
        \addlegendentry{Thread #1}
    }

    \addplot[solid,very thick,white] coordinates {(4,0) (4,{gauss(4)})};
\end{axis}
\end{tikzpicture}

        }
        \begin{alertblock}{Modified code}
            \#pragma omp for schedule(static, size/(2*num\_threads))
        \end{alertblock}
    }
    \pause
\end{frame}

\begin{frame}{Second analysis}
    \begin{columns}
        \begin{column}{.45\linewidth}
            \includegraphics[width=\linewidth]  {tabarnac/is_b_kb2_modif}
        \end{column}
        \begin{column}{.45\linewidth}
            \includegraphics[width=\linewidth]  {tabarnac/is_b_kb1_modif}
        \end{column}
    \end{columns}
\end{frame}

\begin{frame}{Results}
    \includegraphics[width=\linewidth]{tabarnac/is_exectime}
\end{frame}

\begin{frame}{Conclusions}
    \begin{itemize}[<+-|alert@+>]
        \item Collaboration with M. Diener, UFRGS, P.O.A, Brasil
        \item Improved trace collection tool
        \item Simple yet meaningful visualization
        \item Enables significant optimizations
        \item Published at VPA'15~\cite{Beniamine15TABARNAC}
    \end{itemize}
    \uncover<6->{
        \begin{alertblock}{Limitations}
            \begin{itemize}
                \item No temporal informations
                \item Cannot detect precise sharing
            \end{itemize}
        \end{alertblock}
    }
\end{frame}

\section{Collecting and analyzing fine grained traces}

\subsection{Moca an efficient memory trace collection system}

\begin{frame}{Background knowledge}
    \begin{block}{Page faults}
        \begin{itemize}
            \item Mechanism triggered by hardware
            \item Handled by Operation System
            \item Happens at first write
        \end{itemize}
    \end{block}
\end{frame}

\begin{frame}{Design}
    \begin{alertblock}{Linux kernel module}
        \begin{itemize}
            \item Tracks thread creation
            \item Intercept page faults
            \item Injects false page faults
            \item Keep the recent trace (kernel space)
        \end{itemize}
    \end{alertblock}
    \pause
    \begin{block}{User space process}
        \begin{itemize}
            \item Run application with context library
            \item Run application with kernel module loaded
            \item Retrieve trace in user space
        \end{itemize}
    \end{block}
\end{frame}

\begin{frame}{Handling page faults}
    \centering
    \resizebox{!}{.7\textheight}{
        %!TEX encoding=UTF-8 Unicode

\tikzstyle{decision} = [diamond, draw, fill=orange!30,
text width=4.5em, text badly centered, node distance=3cm, inner sep=0pt]
\tikzstyle{mblock} = [rectangle, draw, fill=blue!25,
text width=5em, text centered, rounded corners, node distance=3cm, minimum height=4em]
\tikzstyle{lblock} = [rectangle, draw, fill=blue!10,
text width=5em, text centered, rounded corners, node distance=3cm, minimum height=4em]
\tikzstyle{line} = [draw, -latex']
\tikzstyle{cloud} = [draw, ellipse,fill=orange, node distance=3cm,
minimum height=2em]

\newcommand*{\StrikeThruDistance}{0.15cm}%
%\newcommand*{\StrikeThru}{\StrikeThruDistance,\StrikeThruDistance}%
\tikzset{strike thru arrow/.style={
        decoration={markings, mark=at position 0.5 with {
            \draw [black, very thick,solid,-]
                    ++ (-\StrikeThruDistance,-\StrikeThruDistance)
                    -- ( \StrikeThruDistance, \StrikeThruDistance);
            \draw [black, very thick,solid,-]
                    ++ ( \StrikeThruDistance,-\StrikeThruDistance)
                    -- ( -\StrikeThruDistance, \StrikeThruDistance);
                } },
                    postaction={decorate},
                                    }
        }

    \begin{tikzpicture}[scale=0.75]
        % \draw (0,0) grid (10,-15);
    % Place nodes
        \node [cloud] (OS)   at (0,0){LINUX};
        \node [cloud] (MOCA) at (6,0) {Moca};

        \node [lblock]   (pf)  at (0,-3) {Pagefault (task, @)};
        \node [decision] (mon) at (8,-3) {Is task monitored};

        \node [decision] (shou) at (4,-5) {Should we monitor it};
        \node [lblock]   (hpf)  at (0,-8) {Handle page fault};

        \node [mblock] (Add1) at (4,-8) {Monitor (task)};
        \node [mblock] (Add)  at (8,-8) {Add to chunk (task,@)};

        \node [decision] (sfix)     at (8,-11) {Is it a false page fault};
        \node [lblock]   (resume)   at (0,-14.5){Resume execution};

        \node [mblock]   (fix)      at (8,-14.5) {Fix false page fault (task,@)};

        %% for paths
        \coordinate (inv) at (2,-1);
        \coordinate (inv1) at (2,-8);


    % Draw edges
        \draw  (pf.south) -- ($(pf.south)+(0,-.5)$) -| (inv.south);
        \path [line] (inv.south) -| node [pos=.2,below] {page fault handler} (mon.north);
        \draw [->, strike thru arrow,dashed] (pf) -- (hpf);
        \path [line] (hpf) -- (resume);
        \path [line] (mon.west) -| node [pos=.3,above] {no} (shou.north);
        \path [line] (mon) -- node [left] {yes} (Add);
        \draw (shou.west) -|  node [pos=.75, left] {no} (inv1.south);
        \path [line] (shou) -- node [left] {yes} (Add1);
        \path [line] (Add1) -- (Add);
        \path [line] (Add) -- (sfix);
        \draw  (sfix.west) -| node [pos=.3,above] {no} (inv1.south);
        \path [line] (inv1.south) |- (hpf.east);
        \path [line] (sfix) -- node [left] {yes} (fix);
        \path [line] (fix) -- (resume);
    \end{tikzpicture}


% vim: et si sta lbr  sw=4 ts=4 spelllang=en_us


    }
\end{frame}

\subsection{Experimental evaluation}

\begin{frame}{Tools comparison}
    \small
    \begin{tabular}{lcccc}
        \toprule
         & \textbf{Moca} & \textbf{Tabarnac} & \textbf{Mitos} & \textbf{MemProf} \\
            \midrule
            \textbf{Design} & & & &\\
            \midrule
            Mechanisms   & Page faults  & Inst* & PEBS + Inst* & IBS \\
            Architecture & \textbf{Any} & Intel (AMD) & Intel & AMD   \\
            \midrule
            \textbf{Completeness} & & & &\\
            \midrule
            Granularity & \textbf{Address} & Page          & \textbf{Address} & \textbf{Address} \\
            Superset          & \textbf{Page} & \textbf{Page} & None             & None             \\
            \midrule
            \textbf{Detail} & & & &\\
            \midrule
            Temporal data & \textbf{Yes} & No          & \textbf{Yes} & \textbf{Yes} \\
            CPU location  & \textbf{Yes} & No          & \textbf{Yes} & \textbf{Yes} \\
            Nature        & \textbf{Yes} &\textbf{Yes} & Yes**         & Yes**       \\
        \bottomrule
    \end{tabular}
\end{frame}

\begin{frame}{Benchmarks}
    \small
    \begin{tabular}{lrll}
        \toprule
        \textbf{Name} & \textbf{Footprint*} & \textbf{Description} & \textbf{Group} \\
        \midrule
        \IS & \SI{132}{Mib} & Integer Sort  &
        \multirow{4}{*}{Memory Intensive}\\
        \CG & \si{125}{Mib} & Conjugate Gradient & \\
        \MG & \si{508}{Mib}& Multi-grid & \\
        \FT & \si{398}{Mib}& Discrete 3D FFT & \\
        \midrule
        \UA & \si{112}{Mib}& Unstructured Adaptive mesh &
        \multirow{2}{*}{Unstructured} \\
        \DC & $1.46$Gib & Data Cube & \\
        \midrule
        \BT & \si{120}{Mib}& Block Tri-diagonal solver &
        \multirow{3}{*}{Pseudo Apps} \\
        \SP & \si{122}{Mib}& Scalar Penta-diagonal solver & \\
        \LU & \si{118}{Mib}& Lower-Upper Gauss-Seidel solver & \\
        \midrule
        \EP & \si{78}{Mib}& Embarrassingly parallel & CPU bound\\
        \bottomrule
    \end{tabular}
\end{frame}

\begin{frame}{Experimental results: trace precision}
    \alt<1>{
        \includegraphics[width=\linewidth]{moca/moca_pages_intel.pdf}
    }{
        \includegraphics[width=\linewidth]{moca/moca_addr_intel.pdf}
    }
    \pause
\end{frame}

\begin{frame}{Experimental results: overhead}
    \alt<1>{
        \includegraphics[width=\linewidth]{moca/moca_overhead_intel.pdf}
    }{
        \includegraphics[width=\linewidth]{moca/moca_overhead_amd.pdf}
    }
    \pause
\end{frame}

\subsection{Visualizing and analyzing Moca traces}

\begin{frame}{FrameSoc and Ocelotl}
    \begin{block}{FrameSoc~\cite{Pagano14frameSoC}}
        \begin{itemize}
            \item Generic trace management tool
            \item Several representation of same trace
            \item Designed for trace exploration
        \end{itemize}
    \end{block}
    \pause
    \begin{alertblock}{Ocelotl~\cite{Dosimont14Ocelotl}}
        \begin{itemize}
            \item FrameSoc Tool
            \item Aggregates trace aiming at reducing information loss
            \item Simple zoom and filters operations
        \end{itemize}
    \end{alertblock}
\end{frame}

\begin{frame}{Example: Matrix multiplication}
    \centering
    \includegraphics[width=\linewidth]{ocelotl/overview.png}
\end{frame}

\begin{frame}{Limits}
    \begin{itemize}
        \item FrameSoc trace model not well suited for memory traces
        \item Interaction is too slow
        \item Hard to identify / filter by data structures
    \end{itemize}
\end{frame}

\begin{frame}{Programmatic approach}
    \begin{block}{Idea}
        \begin{itemize}
            \item Programmatic method using R
            \item Orgmod labbook to record our analysis
            \item Basic visualizations inspired from Tabarnac ones
            \item Custom visualizations depending on the trace
        \end{itemize}
    \end{block}
\end{frame}

\begin{frame}{Example: dgetrf}
    \centering
    \includegraphics[width=\linewidth]{labbook/intensity_RW_dgetrf_zoom}
\end{frame}

\section{Conclusions and perspectives}

\begin{frame}{Global Memory analysis}
    \begin{block}{Tabarnac}
        \begin{itemize}
            \item Collaboration with M. Diener
            \item Based on state of the art tool
            \item Results in significant performance gain
            \item Simple to use
            \item Published at VPA'15~\cite{Beniamine15TABARNAC}
        \end{itemize}
    \end{block}
    \pause
    \begin{alertblock}{Limitations}
        \begin{itemize}
            \item No temporal view
            \item Sharing detection is not precise
            \item Patterns analysis is not possible
        \end{itemize}
    \end{alertblock}
\end{frame}

\begin{frame}{Fine grained memory analysis}
    \begin{alertblock}{Moca}
        \begin{itemize}
            \item Based on widely used mechanism
            \item More precise trace than existing tools
            \item Contains all information for pattern detection
            \item Submitted at CCGRID'16
            \item Research reports~\cite{Beniamine16Moca,Beniamine15Memory}
        \end{itemize}
    \end{alertblock}
    \pause
    \begin{block}{Visualization}
        \begin{itemize}
            \item FrameSoc / Ocelotl approach promising but does not scale
            \item R approach permit to visualize actual issues
            \item Still a work in progress
        \end{itemize}
    \end{block}
\end{frame}

\begin{frame}{Perspectives}
    \alertblockat{<1>}{Short term}{
        \begin{itemize}
            \item Use Moca traces to understand Intel MKL performaances
            \item Analyze real application
            \item Couple Moca traces with performances information
        \end{itemize}
    }
    \pause
    \alertblockat{<2>}{Long term}{
        \begin{itemize}
            \item Higher level trace visualization for Moca
            \item Similar tools for GPU / Accelerators
        \end{itemize}
    }
\end{frame}

\newcounter{finalframe}
\setcounter{finalframe}{\value{framenumber}}
%Last numbered frame go here

\begin{frame}{}
    \centering
    \Huge
    Thank You !
\end{frame}
%=============================================================================

%=============================================================================
%Uncomment next lines for uncounted backup slides & biblio
\section*{Bibliography}
%
\bibliographystyle{apalike}
\bibliography{biblio}

%========================= Backup slides =====================================
\section*{Hidden slides}
%put this line before each frame
%\setcounter{framenumber}{\value{finalframe}}

\subsection*{Tabarnac}

\setcounter{framenumber}{\value{finalframe}}
\begin{frame}{Experimental setup}
    \small
    \centering
    \begin{tabular}{lccccc}
        \toprule
        & \multicolumn{5}{c}{\textbf{Hardware totals}}\\
        \cmidrule(lr){2-6}
        & Nodes & Threads & Vendor & Model & Memory \\
        \cmidrule(lr){2-6}
        \texttt{Turing}   & $4$ & $64$ & Intel & Xeon X7550   & \SI{128}{Gib} \\
        \texttt{Idfreeze} & $8$ & $48$ & AMD   & Opteron 6174 & \SI{256}{Gib}\\
        \midrule
        & \multicolumn{5}{c}{\textbf{Hardware per node}}\\
        \cmidrule(lr){2-6}
        & Cores & Threads & Frequency & L3 Cache & Memory \\
        \cmidrule(lr){2-6}
        \texttt{Turing}   & $8$ & $16$ & \SI{2.00}{Ghz}& \SI{18}{Mib} & \SI{32}{Gib} \\
        \texttt{Idfreeze} & $6$ & $6$  & \SI{2.20}{Ghz}& \SI{12}{Mib} & \SI{32}{Gib}\\
        \midrule
        & \multicolumn{5}{c}{\textbf{Software}}\\
        \cmidrule(lr){2-6}
        & Kernel & \multicolumn{2}{c}{Distribution} &
        \multicolumn{2}{c}{Bios configurations} \\
        \cmidrule(lr){2-6}
        \texttt{Turing}   & Linux 3.13 & \multicolumn{2}{c}{Ubuntu 12.04} &
        \multicolumn{2}{c}{Hyper threading} \\
        \texttt{Idfreeze} & Linux 3.2 & \multicolumn{2}{c}{Debian Jessie} &
        \multicolumn{2}{c}{No hyper threading}\\
        \bottomrule
    \end{tabular}
\end{frame}

\setcounter{framenumber}{\value{finalframe}}
\begin{frame}{Tabarnac overhead}
    \includegraphics[width=\linewidth]{tabarnac/tool-ovh.pdf}
\end{frame}

\subsection*{Moca}

\setcounter{framenumber}{\value{finalframe}}
\begin{frame}{Experimental setup}
    \small
    \begin{tabular}{lllllllllll}
        \toprule
        & \multicolumn{5}{c}{\textbf{Hardware totals}}\\
        \cmidrule(lr){2-6}
        & Nodes & Threads & \multicolumn{2}{c}{CPU} & Memory \\
        \cmidrule(lr){2-6}
        \texttt{Edel}    & $2$ & $8$  & \multicolumn{2}{c}{Intel Xeon E5520}      & \SI{24}{Gib} \\
        \texttt{StRemi} & $2$ & $24$ & \multicolumn{2}{c}{AMD Opteron 6164 HE }& \SI{48}{Gib} \\
        \midrule
        & \multicolumn{5}{c}{\textbf{Hardware per node}}\\
        \cmidrule(lr){2-6}
        & Cores & Threads & Frequency & L3 Cache & Memory \\
        \cmidrule(lr){2-6}
        \texttt{Edel}   & $4$  & $4$   & \SI{2.27}{Ghz}& \SI{8}{Mib}  & \SI{12}{Gib} \\
        \texttt{StRemi} & $12$ & $12$  & \SI{1.70}{Ghz}& \SI{12}{Mib} & \SI{24}{Gib}\\
        \midrule
        & \multicolumn{5}{c}{\textbf{Software}}\\
        \cmidrule(lr){2-6}
        & \multicolumn{2}{c}{Distribution} & Kernel &
            \multicolumn{2}{c}{Bios configurations} \\
        \cmidrule(lr){2-6}
        \texttt{Turing}   & \multicolumn{2}{c}{Debian Jessie} & Linux 3.16.0-4 &
            \multicolumn{2}{c}{No hyper threading} \\
        \texttt{Idfreeze} & \multicolumn{2}{c}{Debian Jessie} & Linux 3.16.0-4 &
            \multicolumn{2}{c}{No hyper threading}\\
        \bottomrule
    \end{tabular}
\end{frame}

\setcounter{framenumber}{\value{finalframe}}
\begin{frame}{Parameters evaluation}
   \alt<1>{
    \includegraphics[width=\linewidth]{moca/moca_param.pdf}
    }{
        \includegraphics[width=\linewidth]{moca/moca_param_events.pdf}
    }
    \pause
\end{frame}

\setcounter{framenumber}{\value{finalframe}}
\begin{frame}{Internal design}
    \centering
    \resizebox{!}{.85\textheight}{
        \pgfdeclarelayer{background}
\pgfdeclarelayer{bg1}
\pgfdeclarelayer{foreground}
\pgfsetlayers{background,bg1,main,foreground}

\definecolor{logcolor}{HTML}{FDAE61}
\definecolor{moncolor}{HTML}{FF1922}%FF000A: too dark for print
\definecolor{pfcolor} {HTML}{3B8ECC}
\definecolor{callcolor}{HTML}{FFFFBF}
\definecolor{dtcolorL}{HTML}{ABD9E9}
\colorlet{dtcolor}{dtcolorL!25}

\tikzstyle{handler} = [rectangle, draw, fill=callcolor,
text width=4em, text badly centered]
\tikzstyle{handlerI} = [diamond, text badly centered]
\tikzstyle{data} = [rectangle, draw, fill=dtcolor,
text width=4em, text centered, rounded corners]
\tikzstyle{Chunk} = [rectangle, draw,rounded corners]
\tikzstyle{dataL} = [rectangle, draw, fill=dtcolorL,
text width=4em, text centered, rounded corners]
\tikzstyle{entity} = [draw, ellipse, text centered,
text width=5em, text=white]
\tikzstyle{line} = [very thick,align=center]
\tikzstyle{pf} = [fill=pfcolor,solid]
\tikzstyle{log} = [fill=logcolor,dotted]
\tikzstyle{mon} = [fill=moncolor,dashed]
\tikzstyle{monA} = [-latex,line,dashed,moncolor]
\tikzstyle{pfA} =  [-latex,line,solid,pfcolor]
\tikzstyle{pfAI} =  [line,solid,pfcolor]
\tikzstyle{logA} = [-latex,dotted,line,logcolor]

\tikzset{
    basic box/.style = {
        shape = rectangle,
        draw,
    rounded corners},
    filled box/.style = {
        shape = rectangle,
        draw  = #1,
        fill  = #1,
    rounded corners},
    header node/.style = {
    %Minimum Width = header nodes,
        font          = \strut\large\ttfamily,
        text depth    = +0pt,
        fill          = #1,
    draw},
    header/.style n args={2}{%
        inner ysep = +1.5em,
        append after command = {
            \pgfextra{\let\TikZlastnode\tikzlastnode}
            node [header node=#2] (header-\TikZlastnode) at (\TikZlastnode.north) {#1}
      %node %[span = (\TikZlastnode)(header-\TikZlastnode)]
       % at (fit bounding box) (h-\TikZlastnode) {}
        }
    },
    footer node/.style = {
    %Minimum Width = header nodes,
        font          = \strut\large\ttfamily,
        text depth    = +0pt,
        fill          = #1,
    draw},
    footer/.style n args={2}{%
        inner ysep = +1.5em,
        append after command = {
            \pgfextra{\let\TikZlastnode\tikzlastnode}
            node [header node=#2] (header-\TikZlastnode) at (\TikZlastnode.south) {#1}
      %node %[span = (\TikZlastnode)(header-\TikZlastnode)]
       % at (fit bounding box) (h-\TikZlastnode) {}
        }
    },
    hv/.style = {to path = {-|(\tikztotarget)\tikztonodes}},
    vh/.style = {to path = {|-(\tikztotarget)\tikztonodes}},
    fat blue line/.style = {ultra thick, blue}
}



\begin{tikzpicture}[font=\small,scale=.73, each node/.style={minimum width=1em}]
    % Node placement
    \begin{pgfonlayer}{foreground}
        \node [Chunk] (ch3) at (0,0)        {Chunk3 empty};
        \node [Chunk] (ch2) at (0,-1)       {Chunk2 current};
        \node [Chunk] (ch1) at (0,-2)       {Chunk1 ending};
        \node [Chunk] (ch0) at (0,-3)       {Chunk0 completed};
        \node [Chunk] (cur) at (3,-1.5)   {current};

        \draw [line,->] (cur.north) |- (ch2.east);
    \end{pgfonlayer}

    \node [fit=(cur) (ch0) (ch1) (ch2) (ch3),filled box=dtcolor,
    header={Trace (1 Task)}{dtcolor!50}] (Tr) {};

    \node [data] (PfL) at (8,-1) {False Page fault map};
    \node [data] (TL)  at (12,-1) {Tasks map};


    \uncover<9->{
        \node[handler] (Flh) at (1,-5.5) {Read callback};
    }
    \uncover<2->{
        \node[handler] (PfH) at (10,-4.5) {page fault handler};
    }


    \uncover<8->{
        \node[entity,log] (F) at (1,-8) {Logging Process (userspace)};
    }
    \uncover<5->{
        \node[entity,mon] (M) at (6,-8) {Monitor thread (kernel)};
    }


    \begin{pgfonlayer}{background}
        \node[fit=(F) (M) (cur) (Flh) (PfH) (Tr) (PfL)(TL),header={Moca}{white},basic box] (Moca) {};
    \end{pgfonlayer}

    \node[entity,text width=5.5em,pf] (T) at (12,-14) {Tasks\\(user/kernel\\thread/process)};

    \node [dataL] (pgt)  at (6,-14)     {Page table (kernel)};
    \node [dataL] (proc) at (2,-14)     {/proc file (kernel)};
    \node [dataL] (file) at (-2,-14)     {File (userspace)};

    \begin{pgfonlayer}{background}
        \node[fit=(file) (proc) (pgt) (T),footer={Linux}{white} ,basic box=red,
        %below=1.5em of Moca
        ] (Linux) {};
    \end{pgfonlayer}

    % Edges

    %% Loging process
    \uncover<8->{
        \path[logA] (F.south) edge[in=140,out=270] node[right] {read} (proc.west);
    }
    \uncover<9->{
        \path[logA] (proc.north) edge[in=-20,out=70]  node[pos=.2,left] {triggers} (Flh.east);
    }
    \uncover<10->{
        \path[logA] (Flh.north) edge[out=90,in=270] node[pos=.2,left] {read write} (ch0.south);

        \path[logA] (Flh.west) edge[in=100,out=230] node[pos=.1,left] {write} (file.north);
    }

    % Page faults
    \uncover<2->{
        \path[pfA] (T)   edge[out=70,in=-70] node[pos=.2,right] {triggers} ($(PfH.south)+(1,0)$);
        \coordinate (inv2) at ($(M.south)-(0,0.5)$);
        \path[pfA] (PfH.north) edge node[right] {read \\ (write)} (TL.south);
    }

    \uncover<3->{
        \coordinate (inv) at (8.5,0.5);
        \path[pfAI] (PfH.north) edge[out=100,in=-10] (inv);
        \path[pfA]  (inv)     edge[out=170,in=20] node[pos=.1,above] {write} ($(ch2.east)+(0,0.2)$);
    }

    \uncover<4->{
        \path[pfA] (PfH.south) edge[out=-90, in=90] node[pos=.8,right] {read / write}(pgt.north);
        \path[pfA] (PfH.north) edge[out=160,in=310] node[pos=.8,right] {read} ($(PfL.south)+(0.2,0)$);

    }

    %Monitor
    \uncover<5->{
        \path[monA] (M.north) edge[out=110,in=-30] node[pos=.6,left] {write} (cur.east);
    }
    \uncover<6->{
        \path[monA] (M.north) edge[out=170,in=-10] node[pos=.5,left] {read} (ch1.east);
    }
    \uncover<7->{
        \path[monA] (M.north) edge[out=70,in=290] node[pos=.2,right] {write} (PfL.south);
        \path[monA] (M)  edge[out=-130,in=110] node[pos=.8,left] {write} ($(pgt.north)-(.6,0)$);

    }
\end{tikzpicture}


    }
\end{frame}

\setcounter{framenumber}{\value{finalframe}}
\begin{frame}{Matrix multiplication full example}
    \centering
    \alt<1>{
        \resizebox{.8\linewidth}{!}{
            % -
 %              DO WHAT THE FUCK YOU WANT TO PUBLIC LICENSE
 %                      Version 2, December 2004
 %   
 %   Copyright (C) 2016 Beniamine, David <David@Beniamine.net>
 %   Author: Beniamine, David <David@Beniamine.net>
 %   
 %   Everyone is permitted to copy and distribute verbatim or modified
 %   copies of this license document, and changing it is allowed as long
 %   as the name is changed.
 %   
 %              DO WHAT THE FUCK YOU WANT TO PUBLIC LICENSE
 %     TERMS AND CONDITIONS FOR COPYING, DISTRIBUTION AND MODIFICATION
 %   
 %    0. You just DO WHAT THE FUCK YOU WANT TO.
 %%

%!TEX encoding=UTF-8 Unicode
%Palette YlOrRd, 3 col
\definecolor{ColV}{HTML}{FC8D59}
\definecolor{ColH}{HTML}{D7301F}

%Palette 4-class paired
\definecolor{Col0}{HTML}{A6CEE3}
\definecolor{Col1}{HTML}{1F78B4}
\definecolor{Col2}{HTML}{B2DF8A}
\definecolor{Col3}{HTML}{33A02C}


\pgfdeclarelayer{background}
\pgfdeclarelayer{foreground}
\pgfsetlayers{background,foreground}


\tikzstyle{PrimaryA}   = [-latex,very thick]
\tikzstyle{SecondaryA} = [-latex,very thick,dashed]
\tikzstyle{SwapA} = [latex-latex, thick,dotted]

\newcommand{\coli}[1]{\textcolor{ColI}{#1}}
\newcommand{\colj}[1]{\textcolor{ColJ}{#1}}
\newcommand{\colk}[1]{\textcolor{ColK}{#1}}

\newcommand{\Ta}{\textcolor{Col0}{Thread~1}\xspace}
\newcommand{\Tb}{\textcolor{Col1}{Thread~2}\xspace}
\newcommand{\Tc}{\textcolor{Col2}{Thread~3}\xspace}
\newcommand{\Td}{\textcolor{Col3}{Thread~4}\xspace}

\tikzset{
    algorithm/.style={
        shape=rectangle,
        alias=this,
        append after command = {
            \pgfextra{
              % Top and bottom lines
                \draw [] ($(this.north west)+(0,.5)$) -- ($(this.north east)+(0,.5)$);
                \node [anchor=west] at ($(this.north west)+(0,0.25)$) {\textbf{Algorithm} #1};
                \draw [] (this.north west) -- (this.north east);
                \draw [] (this.south west) -- (this.south east);
            }
        }
    },
    matgrid/.style args={#1#2}{
        %#1: name #2: size
        alias=this,
        append after command ={
            \pgfextra{
                \draw[very thin,loosely dotted,step=.5] (this) grid ($(this)+(#2,#2)$);
                \draw (this) rectangle ($(this)+(#2,#2)$);
                % Four corners
                \coordinate (m#1-00) at ($(this)+(0,0)$);
                \coordinate (m#1-0N) at ($(this)+(0,#2)$);
                \coordinate (m#1-N0) at ($(this)+(#2,0)$);
                \coordinate (m#1-NN) at ($(this)+(#2,#2)$);

                \coordinate(m#1-east) at ($(m#1-00)!.5!(m#1-0N)$);
                \coordinate(m#1-west) at ($(m#1-N0)!.5!(m#1-NN)$);
                \coordinate(m#1-north)  at ($(m#1-00)!.5!(m#1-N0)$);
                \coordinate(m#1-south)  at ($(m#1-0N)!.5!(m#1-NN)$);

                \node (#1) at ($(m#1-00)+(-.5,0)$){\textbf{#1}};

            }
        }
    }
}



\begin{tikzpicture}[font=\small]

    \begin{pgfonlayer}{background}

        \node[matgrid={A}{4}] at (0,0){};
        \node[matgrid={B}{4}] at (5,5){};
        \node[matgrid={C}{4}] at (5,0){};

    \end{pgfonlayer}

    % Indexes

    \begin{pgfonlayer}{foreground}
        %% A
        \draw[very thick,ColV] (mA-west) -- (mA-east);
        \draw[very thick,ColV] (mC-west) -- (mC-east);
        \draw[very thick,ColH] (mB-north) -- (mB-south);
        \draw[very thick,ColH] (mC-north) -- (mC-south);

        \node at ($(mA-east)!0.5!(mA-west)+(0,1)$) {\Ta and \Tb};
        \node at ($(mA-east)!0.5!(mA-west)+(0,-1)$) {\Tc and \Td};
 
        \node[text centered,text width=5em] at ($(mB-north)!0.5!(mB-south)+(-1,0)$) {\Ta and \Tc};
        \node[text centered,text width=5em] at ($(mB-north)!0.5!(mB-south)+(1,0)$) {\Tb and \Td};

        \node[text centered] at ($(mC-east)!0.5!(mC-west)+(-1,1)$)  {\Ta};
        \node[text centered] at ($(mC-east)!0.5!(mC-west)+(1,1)$)   {\Tb};
        \node[text centered] at ($(mC-east)!0.5!(mC-west)+(-1,-1)$) {\Tc};
        \node[text centered] at ($(mC-east)!0.5!(mC-west)+(1,-1)$)  {\Td};
    \end{pgfonlayer}

\end{tikzpicture}
% vim: et si sta lbr  sw=4 ts=4 spelllang=en_us

        }
    }{
        \alt<2>{
            \includegraphics[width=\textwidth]{ocelotl/TaskView.png}
        }{
            \alt<3>{
                \includegraphics[width=\textwidth]{ocelotl/TaskView-zoom-init.png}
            }{
                \alt<4>{
                    \includegraphics[width=\textwidth]{ocelotl/Sharing.png}
                }{
                    \includegraphics[width=\textwidth]{ocelotl/Sharing-zoom.png}
                }
            }
        }
    }
    \pause
    \pause
    \pause
    \pause
\end{frame}%
\setcounter{framenumber}{\value{finalframe}}
\begin{frame}{dgetrf full example}
    \centering
    \alt<1>{
        \includegraphics[width=\textwidth]{labbook/intensity_Rw_dgetrf}
    }{
        \alt<2>{
            \includegraphics[width=\textwidth]{labbook/intensity_RW_dgetrf_zoom}
        }{
            \alt<3>{
                \includegraphics[width=\textwidth]{labbook/intensity_Share_dgetrf_zoom-init}
            }{
                \includegraphics[width=\textwidth]{labbook/intensity_Share_dgetrf_zoom-init1}
            }
        }
    }
    \pause
    \pause
    \pause
\end{frame}%
%=============================================================================
\end{document}
