%!TEX encoding=UTF-8 Unicode


I first would like to thank the members of my jury for the time they spent on my work and for their helpful remarks.

I would like to thank all the members of the Informatica team of Porto Alegre for welcoming me there.
Especially Mathias and Marcos, I enjoyed working with you guys, it has been a great experience.

Traduire en portugais:
Je voudrais aussi remercier tous mes amis à POA qui m'ont accueilli et qui ont éclaircit mon séjour.
Plus précisément je voudrais remercier tous les habitant de Riachuelo et les visiteurs réguliers pour toutes les soirées et discussions passionnantes.

Je tiens à remercier tous les membres présents et passés des équipes Polaris/Moais/Datamove/Mescal, travailler avec vous à été très enrichissant, scientifiquement comme humainement.
Et surtout merci à Annie Simon et Christine Guimet qui plus d'une fois ont rattrapé mes bêtises administratives.
Je voudrais aussi remercier les enseignants de l'UFR IM2AG qui m'ont motivé à pousser les études jusqu'à la thèse et ont continué à suivre mon évolution, en particulier Anne Rasse.
Dans un autre registre, tous les coincheur.euse.s grâce à qui les pauses midi finissent souvent en fou rire et concours de mauvaise foi.
Tous mes co-bureaux même temporaires qui ont rendu les heures de déprime de rédaction ou de debug supportables et souvent même agréables.
Plus particulièrement Raphaël (avec ou sans son genou) et Alexis pour tous les interminables débats mi-Trolls mi-sérieux et les croissants du lundi.
Swann qui rentre à la fois dans toutes ces catégories et dans aucune, et sans qui je n'aurais peut-être pas atterri dans ce laboratoire.
Enfin je voudrais remercier mes encadrants et surtout Guillaume pour ces 5 années durant lesquelles il m'a accompagné et encouragé dans mon travail, qu'il s'agisse de la thèse ou d'autres projets, toujours avec beaucoup d'humour, parfois même sur un vélo ou derrière un verre (heureusement jamais les deux à la fois).
Bref j'ai eu de la chance d'avoir un aussi bon encadrant\footnote{
    “le mauvais encadrant, il voit un thésard, il l'encadre, alors que le bon encadrant, il voit un thésard, il l'encadre, mais c'est un bon encadrant\ldots”
}, j'espère vraiment que nous continuerons à travailler ensemble.

Je voudrais finalement remercier tou.te.s mes ami.e.s qui depuis des années me supportent avec tous mes défauts et sont toujours présent malgré tout.
Tous les sportifs occasionnels ou réguliers, sans nos sorties vélo, via, rando, etc. je serais probablement devenu fou.
Parmi eux je tiens à remercier spécifiquement Seb pour sa capacité à transformer n'importe quelle “petite sortie d'une heure” en aventure, parfois en y laissant une roue, des sacoches ou carrément un genou (encore un).
Je tiens aussi à remercier mes parents et Sacha pour avoir subi mon incapacité à organiser ma vie hors de la thèse sans jamais se plaindre ni m'en tenir rigueur mais aussi pour leurs encouragements.
Enfin, Flo, je ne peux pas imaginer ce qu'auraient été ces dernières années ni comment aurait fini cette thèse sans toi, je ne peux pas non plus te dire tout ce pour quoi je veux te remercier sans doubler (au minimum) la taille de ce manuscrit, alors je dirais juste: “thanks for all the fish” (mais pas “so long”, bien au contraire).

\glsresetall
% vim: et si sta lbr  sw=4 ts=4 spelllang=fr
