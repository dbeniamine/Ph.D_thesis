%!TEX encoding=UTF-8 Unicode
\chapter{Memory Performance Analysis}

The results of our case study (\chap{perf}) showed that traditional performance analysis tools can help identify memory related performance issues.
Yet they are not able to tell precisely where, in terms of data structures, the issue occurs and thus it is still required to analyze the code manually.
As memory is often a performance bottleneck, several tools where developed to analyze performance in regards of the memory.

This chapter discuss memory analysis tools, first we present the specificities of recent memory subsystems and the usual mistakes that can generate performances drop in \sect{archi}.
Finally we present existing performance analysis tools related to memory in \sect{mem-tools} and describe how we could improve these tools in \sect{mem-cncl}.

\section{Architectural considerations}
\label{sec:archi}

Since a few decades, processor frequency increased significantly more than memory frequencies, resulting in a considerable gap between these two resources.
Indeed, retrieving one piece of data cost around $100$ \gls{CPU} cycles~\cite{Drepper07What}.
To mitigate this issue, the \glspl{CPU} embed a cache hierarchy that allow to keep highly used data closer to them reducing the access time to around $10$ cycles.
These caches comes with several mechanisms to optimize their usage, yet a code written without considering memory pattern can easily defeat all these optimizations resulting in considerable performance lost.

\begin{figure}[htb]
    \centering
    %!TEX encoding=UTF-8 Unicode
%Palette GnBu 5 col + white
\definecolor{ColPU}{HTML}{FFFFFF}
\definecolor{ColCore}{HTML}{F0F9E8}
\definecolor{ColL1}{HTML}{BAE4BC}
\definecolor{ColL2}{HTML}{7BCCC4}
\definecolor{ColL3}{HTML}{43A2CA}
\definecolor{ColM}{HTML}{0868AC}
\definecolor{ColS}{HTML}{FFFFFF}

\pgfdeclarelayer{bg}
\pgfdeclarelayer{bbg}
\pgfdeclarelayer{bbbg}
\pgfsetlayers{bbbg,bbg,bg,main}


\tikzset{
    box/.style={
        shape=rectangle,
        text centered,
        draw,
    },
    pics/core/.style args={#1#2#3#4}{
        % Args: #1: nb PU, #2 core id, #3 direction: + or -
        code={
            \pgfmathtruncatemacro{\pmin}{#1*#2}
            \pgfmathtruncatemacro{\pmax}{\pmin+#1-1}
            %PUs
            \foreach \pu in {\pmin,...,\pmax}{
                \pgfmathtruncatemacro{\pstep}{\pu-\pmin}
                \node[box,fill=ColPU] (PU-\pu)at (0,#3.7*\pstep) {PU\#\pu};
            }
            %Core ID
            \node (inv) at ($(PU-\pmin)#3(0,-.3)$) {};
            \node[minimum width=3.3em] (name) at ($(PU-\pmin)!0.5!(PU-\pmax)#3(0,1)$) {Core\##2};
            % L1
            \begin{pgfonlayer}{bg}
                \node[box,fill=ColCore,inner sep=.1pt, fit=(name) (PU-\pmin)
                (PU-\pmax) (inv)] (core-#2) {};
            \end{pgfonlayer}
            \draw[fill=ColL1] ($(core-#2.#4 west)#3(0,.1)$) rectangle ($(core-#2.#4 east)#3(0,.5)$)%
            node[pos=.5] (l1) {L1};
            % links
            \draw (core-#2.#4) -- ($(core-#2.#4)#3(0,.1)$);
            \coordinate (l1-#2-n) at ($(core-#2.#4)#3(0,.5)$);
        }
    },
    pics/l2group/.style args={#1#2#3#4}{
        % Args: #1: nb Cores, #2 group id, #3 direction: + or -
        code={
            \pgfmathtruncatemacro{\cmin}{#1*#2}
            \pgfmathtruncatemacro{\cmax}{\cmin+#1-1}
            % Cores
            \foreach \core in {\cmin,...,\cmax}{
                \pgfmathtruncatemacro{\cstep}{\core-\cmin}
                \draw (1.4*\cstep,0) pic[font=\tiny] {core={2}{\core}{#3}{#4}};
            }
            % L2
            \draw[fill=ColL2] ($(core-\cmin.#4 west)#3(0,.6)$) rectangle
            ($(core-\cmax.#4 east)#3(0,1.1)$) node[pos=.5]{L2};
            % Coordinates for L3
            \coordinate (l2g-#2-w) at ($(core-\cmin.#4 west)#3(0,1.1)$);
            \coordinate (l2g-#2-e) at ($(core-\cmax.#4 east)#3(0,1.1)$);
            %links
            \foreach \core in {\cmin,...,\cmax}{
                \draw (l1-\core-n) -- ($(core-\core.#4)#3(0,.6)$);
            }
        }
    },
    pics/socket/.style args={#1#2#3#4}{
        % Args: #1: nb l2 groups, #2 socket id, #3 arguments for underlying pics
        code={
            \pgfmathtruncatemacro{\lmin}{#1*#2}
            \pgfmathtruncatemacro{\lmax}{\lmin+#1-1}
            % Cores
            \foreach \lg in {\lmin,...,\lmax}{
                \pgfmathtruncatemacro{\lgstep}{\lg-\lmin}
                \draw (2.8*\lgstep,0) pic[font=\small] {l2group={2}{\lg}{#3}{#4}};
            }
           % L3
            \draw[fill=ColL3] ($(l2g-\lmin-w)#3(0,.1)$) rectangle ($(l2g-\lmax-e)#3(0,.5)$)%
            node[pos=.5]{L3};
           % Coordinates for Mem
            \coordinate (s-#2-w) at ($(l2g-\lmin-w)#3(0,.5)$);
            \coordinate (s-#2-e) at ($(l2g-\lmax-e)#3(0,.5)$);
           % links
           %\foreach \lg in {\lmin,...,\lmax}{
           %    \draw ($(l2g-\lg-w)!.5!(l2g-\lg-e)$) --
           %        ($(l2g-\lg-w)!.5!(l2g-\lg-e)#3(0,.1)$);
           %}
           % CPU
            \ifthenelse{\equal{#3}{+}}{
                \node (minnode) at (-.6,-.3) {};
            }{
                \node (minnode) at (-.6,.3) {};
            }
            \node (maxnode) at ($(l2g-\lmax-e)#3(0,1)$) {};
            \node (sockname) at ($(l2g-\lmin-e)#3(0,.8)$) {Socket \##2};

            \begin{pgfonlayer}{bbg}
                \node[box,fill=ColS,fit=(minnode) (maxnode)] (cpu-#2) {};
            \end{pgfonlayer}
           % Memory
            \draw[fill=ColM,text=white] ($(cpu-#2.#4 west)#3(0,.5)$) rectangle ($(cpu-#2.#4 east)#3(0,1.5)$)%
            node[pos=.5]{Memory bank \##2};

            \draw[very thick] (cpu-#2.#4) -- ($(cpu-#2.#4)#3(0,.5)$);

        }
    },
}

\begin{tikzpicture}[font=\small, every pic/.style={scale=.9}]
    \pic at (0,0)  {socket={2}{0}{+}{north}};
    \pic at (6.5,0)  {socket={2}{1}{+}{north}};
    \pic at (0,-2) {socket={2}{2}{-}{south}};
    \pic at (6.5,-2) {socket={2}{3}{-}{south}};

    \begin{pgfonlayer}{bbbg}
        \draw[line width=1em] (cpu-0) -- (cpu-2);
        \draw[line width=1em] (cpu-0) -- (cpu-1);
        \draw[line width=.3em]  (cpu-0) -- (cpu-3);
        \draw[line width=1em] (cpu-1) -- (cpu-3);
        \draw[line width=1em] (cpu-2) -- (cpu-3);
        \draw[line width=.3em]  (cpu-1) -- (cpu-2);
    \end{pgfonlayer}

\end{tikzpicture}
% vim: et si sta lbr  sw=4 ts=4 spelllang=en_us

    \caption{An simple example of NUMA topology}
    \label{fig:topo-NUMA}
\end{figure}

To understand these mechanisms and how to benefit from them, we will consider hypothetical \gls{NUMA} machine represented in \fig{topo-NUMA}.
Although the parallelism inside one \gls{CPU} socket keeps increasing, it is easier and more economical to double the number of socket inside one machine than the number of cors inside one socket.
When building a machine with several sockets we can either give them a uniform access to the memory by sharing the memory bus or split the memory into banks and giving non uniform access to the sockets.
While the first option seems simpler to use, it means that the bandwidth is shared by all the threads, therefore contention can easily appear.
At the opposite, the second option provides a maximal bandwidth for each socket.
Still, writing code that uses efficiently this specific architecture remains the burden of the developer who therefore needs to explicitly consider the physical location of its data.

From the \gls{OS} point of view, the memory is split into contiguous chunks called pages, usually one page correspond to \SI{4}{Kib}.
This pagination is used to provide the abstraction of virtual memory.
Indeed, userspace programs uses virtual pages which are not actually mapped to the physical memory until it is absolutely necessary to do so.
Linux is a lazy \gls{OS} thus it will never map a page until a program has wrote something on it.
Indeed if a program reads the contents of a new page, Linux can simply return a zero.
To do so, one full of zeroes is always present in the memory and any virtual page points to this specific page until a program write it.
This means that the physical location of a piece of data is determined the first time that a program write something on the page on which it has been allocated.
Furthermore, the page will be mapped according to the first touch~\cite{Marchetti95Using} policy.
This policy maps a page on the closest memory bank to the socket where the thread responsible for the first access is executing.

As a result, if two independents data structures are allocated on the same page they will be mapped on the same memory bank.
Moreover, this mapping will depend on the \gls{CPU} location of the thread initializing this page.
A classic performance issue with \gls{NUMA} machines consist in initializing all the data structure with only one thread.
When doing so all pages are mapped on the same memory bank, and each access from another socket will be remote thus slow.
An easy way to overpass this issue for small computational kernels consists in running a loop of computation on the data before initializing them.
Indeed, by doing so each page will be mapped as close as possible to the first thread that will actually used it.
Yet, this approach is more a hack than a real solution and is not suitable for larger programs.
Kleen et al. developed an interface to map the pages on \gls{NUMA} machines~\cite{Kleen05NUMA}.
This \gls{API} can be accessed either via the \texttt{numactl} command or via a library called \texttt{libnuma}.
The \texttt{numactl} command is useful to apply a global policy on all the page of the application.
It is often use to apply the \emph{interleave} policy that distribute the pages over the \gls{NUMA} nodes in a round robin way.
While this policy does not reduce the overall number of remote accesses, it distribute them among the nodes and therefore reduces the contention when there are more than two sockets.
At the opposite the \texttt{libnuma} provides fine grain page and thread mapping.
The user can use it to specify explicitly allocate data structures and bind thread on nodes.
Still finding the optimal mapping for one machine not trivial.
Furthermore mapping threads and data structures in an adaptive way is even harder.
Therefore several tools were developed to automatically map threads and pages online~\cite{Diener14kMAF,Corbet12Toward}.

Each socket of our hypothetical machine is composed of three level of caches as we can see in \fig{topo-NUMA}.
When a thread need to access a piece of data, it will look for it in its L1 cache if the data is not there (cache miss), it will go to the next level until we reach the main memory.
Caches works at the granularity of the \emph{cache line}, usually \SI{64}{bytes}.
If a data is not in the cache, it will retrieve a whole line from the main memory.
Inserting a new line mean evicting an other one.
Finding the optimal cache line to evict is the on that will be used in the longest time, obviously this is impossible as it requires to predict the future.
Usually caches remove the \acrfull{LRU} line.
To avoid to check the age of every line each time a line is evicted, caches are $N$-way associative which means that each line can only go in $N$ different position in the cache.
This associativity can be used at the allocation time to partition the cache and allocate more cache to the data structure that will benefit from it~\cite{Perarnau11Controlling}.

Each line present in a cache level is also present in all the caches above in the hierarchy.
A conflicts occurs when two threads write the same line of code, even if they are note writing the same part of the line.
Such conflicts are resolved at the lower level of cache common to both threads.
Therefore the farther the threads are, the more costly it will be to solve the conflict.
Thus to optimize the performances of an application we must keep the threads working on the same data as close as possible.

Finally, the cache prefecther try to detect memory access pattern to retrieve several line of cache at the same time from the main memory.
This mechanism is particularity efficient with linear accesses.
Yet, for sparse accesses it might evict more cache line then necessary.

\begin{figure}[htb]
    \centering
    %!TEX encoding=UTF-8 Unicode
%Palette PurOr 4 Col
\definecolor{ColG}{HTML}{5E3C99}
\definecolor{ColB}{HTML}{FDB863}

\newcommand{\colg}[1]{\textcolor{ColG}{#1}}
\newcommand{\colb}[1]{\textcolor{ColB}{#1}}

\pgfdeclarelayer{bg}
\pgfsetlayers{bg,main}


\tikzstyle{arr}   = [-latex,thick]
\tikzstyle{txtnode} = [anchor=west]
\tikzstyle{mybrace} = [decorate,decoration={brace, mirror,amplitude=1em},thick]
\tikzstyle{mydash} = [dashed, dash pattern=on 1pt off 2pt]

\def\eltsz{3}
\def\vshift{-1.5}
\def\balign{.6}
\pgfmathparse{\balign*\eltsz}
\edef\balignsz{\pgfmathresult}
\def\nlines{1}
\pgfmathparse{\nlines-\balign}
\edef\resid{\pgfmathresult}

\tikzset{
    myarray/.style args={#1#2#3#4}{
        % args: size-1, fill, show number ?
        alias=this,
        append after command = {
            \pgfextra{
                \coordinate (#4-base) at ($(this.west)+(0,\vshift)$);
                \pgfmathparse{#1-1}
                \foreach \i in {0,...,\pgfmathresult}{
                    \draw[fill=#2] ($(#4-base)+(\eltsz*\i,0)$) rectangle ($(#4-base)+(\eltsz*\i+\eltsz,1)$);
                    \ifthenelse{#3=0}{}{
                        \node[anchor=south] at ($(#4-base)+(\eltsz*\i,1)$) {\i};
                    }
                }
                \ifthenelse{#3=0}{}{
                    \node[anchor=south] at ($(#4-base)+(\eltsz*#1,1)$){#1};
                }
            }
        }
    },
}



\begin{tikzpicture}[font=\small]
    \node[myarray={4}{none}{1}{bad},txtnode] (bad) at (0,0) {\colb{\textbf{Bad alignment:}}};

    \begin{pgfonlayer}{bg}
        \node[myarray={2}{ColG,mydash}{0}{invb},txtnode] (d0) at (\balignsz,0) {};
        \path[pattern=north east lines, pattern color=ColB] (0,\vshift) rectangle ($(0,\vshift)+(\balignsz,1)$);
        \path[pattern=north east lines, pattern color=ColB] (\balignsz+\eltsz*2,\vshift) rectangle (3*\eltsz,1+\vshift);
    \end{pgfonlayer}

    \draw[mybrace] (0,\vshift) -- (1*\eltsz,\vshift) node [midway,below=1em]
        {Fetch 0};
    \draw[mybrace] (\eltsz,\vshift) -- (2*\eltsz,\vshift) node [midway,below=1em]
        {Fetch 1};

    \draw[mybrace] (2*\eltsz,\vshift) -- (3*\eltsz,\vshift) node [midway,below=1em]
        {Fetch 2};

    \node[anchor=west] at (0,2*\vshift)
    {\textbf{Total:} \colb{3 fetches}, \colg{2~useful lines} / \colb{1~useless line}};

    %% Good alignment
    \node[myarray={4}{none}{1}{good},txtnode] (good) at (0,3*\vshift) {\colg{\textbf{Good alignment:}}};

    \begin{pgfonlayer}{bg}
        \node[myarray={2}{ColG,mydash}{0}{invg},txtnode] (d1) at (0,3*\vshift) {};
    \end{pgfonlayer}

    \draw[mybrace] (0,\vshift+3*\vshift) -- (\eltsz,\vshift+3*\vshift) node
        [midway,below=1em] {Fetch 0};
    \draw[mybrace] (\eltsz,\vshift+3*\vshift) -- (2*\eltsz,\vshift+3*\vshift) node
        [midway,below=1em] {Fetch 1};
    \node[anchor=west] at (0,5*\vshift)
    {\textbf{Total:} \colg{2 fetches}, \colg{2~useful lines} / \colb{0~useless lines}};

\end{tikzpicture}
% vim: et si sta lbr  sw=4 ts=4 spelllang=en_us

    \caption[Example of Bad alignment]{Retrieving two lines of cache with one or two fetches depending on the alignment of the lines.}
    \label{fig:bad-align}
\end{figure}

To benefit from these mechanisms, it is crucial to align our data structure to the cache lines.
\fig{bad-align} show a simple example of badly aligned data structure with a small array of 2 cache lines.
We can see that in the first case, it requires two fetches from the memory to retrieve the whole data structure.
Furthermore we retrieve four lines of cache we only two are required.
Thus we introduced two useless lines in our cache that could have been spared for meaningful data.
At the opposite, in the second case where the data structure is correctly aligned, we retrieve the whole structure with one fetch and does not introduce any useless data in the cache.

\begin{figure}[htb]
    \centering
    %!TEX encoding=UTF-8 Unicode

\definecolor{Col0}{HTML}{1B9E77}
\definecolor{Col1}{HTML}{D95F02}

%\newcommand{\col0}[1]{\textcolor{Col0}{#1}}
%\newcommand{\col1}[1]{\textcolor{Col1}{#1}}

\pgfdeclarelayer{bg}
\pgfsetlayers{bg,main}


\tikzstyle{arr}   = [-latex,very thick]
\tikzstyle{txtnode} = [anchor=west]

\def\eltsz{1.5}
%\def\balign{.4}
%\def\nlines{2}
%\pgfmathparse{\nlines-\balign}
%\edef\resid{\pgfmathresult}

\tikzset{
    myarray/.style args={#1#2}{
        % args: size-1
        alias=this,
        append after command = {
            \pgfextra{
                \coordinate (base) at (this.west);
                \foreach \i in {0,...,#1}{
                    \draw[fill=#2] ($(base)+(\eltsz*\i,0)$) rectangle ($(base)+(\eltsz*\i+\eltsz,1)$);
                }
            }
        }
    },
}



\begin{tikzpicture}[font=\small]
    \node[myarray={7}{none}] (cache) at (0,0) {};
    \draw[arr,Col0] (.5,.5) -- node[above,pos=.5] {Thread 0} (4*\eltsz-.5,.5);
    \draw[arr,Col1] (4*\eltsz+.5,.5) -- node[above,pos=.5] {Thread 1} (8*\eltsz-.5,.5);
\end{tikzpicture}
% vim: et si sta lbr  sw=4 ts=4 spelllang=en_us

    \caption[Example of false sharing.]{Two threads writing 4 consecutive doubles on the same line of cache, without any actual sharing, resulting in false sharing and an easy fix by padding the data structure.}
    \label{fig:false-sharing}
\end{figure}

Yet aligning correctly our data structures is not enough when for parallel code.
Indeed if we  consider a simple example where two threads are working on a small array of $8$ doubles.
As a double is usually coded on $8$ bytes, this array is exactly the size of one cache line.
In this example the computation on each value of the array is independent thus we can divide it in two and each thread will  work on one half as illustrated in \fig{false-sharing}.
The first access will copy the whole array from the memory to all the caches of the thread that triggers it.
When the second thread reads the array, it will copy it from the lower common cache to its private cache.
If the threads are only reading it, no more memory access is required and the performance will be optimal.
Yet in a more realistic example, each thread will update the value of each entry of its array.
Every time a thread writes a value of the array, it invalidate the whole line.
Therefore the coherency protocol must interfere at the lower common level of cache between the two threads.
In the end of the day, each time a thread writes a value, the other one needs to update its line from the L2 or L3 cache, while the two threads are not actually sharing any data thus the name false sharing.
Not only the accesses to this array are inefficient but it generate lot of useless data traffic in the caches bus which can create some contention slowing down the whole application.
The easiest way to fix such issues, consist in padding the data structure, which mean introducing zeros in a way that each thread works on a different cache line.

\begin{figure}[htb]
    \centering
    %!TEX encoding=UTF-8 Unicode

\definecolor{ColI}{HTML}{1B9E77}
\definecolor{ColK}{HTML}{D95F02}
\definecolor{ColJ}{HTML}{7570B3}

\pgfdeclarelayer{background}
\pgfdeclarelayer{foreground}
\pgfsetlayers{background,foreground}


\tikzstyle{PrimaryA}   = [-latex,very thick]
\tikzstyle{SecondaryA} = [-latex,very thick,dashed]

\newcommand{\coli}[1]{\textcolor{ColI}{#1}}
\newcommand{\colj}[1]{\textcolor{ColJ}{#1}}
\newcommand{\colk}[1]{\textcolor{ColK}{#1}}

% #1: name, #2: start pos, #3: size
\newcommand{\matgrid}[3]{
        \draw #2 grid ($#2+(#3,#3)$);
        % Four corners
        \coordinate (m#1-00) at ($#2+(0.5,0.5)$);
        \coordinate (m#1-0N) at ($#2+(0.5,#3-0.5)$);
        \coordinate (m#1-N0) at ($#2+(#3-0.5,0.5)$);
        \coordinate (m#1-NN) at ($#2+(#3-0.5,#3-0.5)$);

        \node (#1) at ($(m#1-00)+(-1,0)$){#1};

        \node (#1) at ($(m#1-0N)+(-.2,.2)$)  {0};
        \node (#1) at ($(m#1-00)+(0,-.2)$) {N-1};
        \node (#1) at ($(m#1-NN)+(.2,0)$)  {N-1};
}


\begin{tikzpicture}[font=\small]

    \begin{pgfonlayer}{background}
        \node[draw,rounded corners] at (2.5,9){%
            \begin{varwidth}{\linewidth}
                \begin{algorithmic}
                    \For{\coli{i in 0..N-1}}
                    \For{\colj{j in 0..N-1}}
                    \For{\colk{k in 0..N-1}}
                    \State C[\coli{i},\colj{j}] += A[\coli{i},\colk{k}] * B[\colk{k},\colj{j}]
                            \EndFor
                        \EndFor
                    \EndFor
                \end{algorithmic}%
            \end{varwidth}%
        };

        \matgrid{A}{(0,0)}{5}
        \matgrid{B}{(6,6)}{5}
        \matgrid{C}{(6,0)}{5}

    \end{pgfonlayer}

    % Indexes

    \begin{pgfonlayer}{foreground}
        %% A
        \draw[PrimaryA,ColI]   (mA-0N) -- node [above] {i} (mA-NN);
        \draw[SecondaryA,ColK] (mA-0N) -- node [left]  {k} (mA-00);

        %% B
        \draw[PrimaryA,ColJ]   (mB-0N) -- node [left]  {j} (mB-00);
        \draw[SecondaryA,ColK] (mB-0N) -- node [above] {k} (mB-NN);

        %% C
        \draw[PrimaryA,ColI]   (mC-0N) -- node [above] {i} (mC-NN);
        \draw[SecondaryA,ColJ] (mC-0N) -- node [left]  {j} (mC-00);
    \end{pgfonlayer}

\end{tikzpicture}
% vim: et si sta lbr  sw=4 ts=4 spelllang=en_us

    \caption{Example of non linear memory accesses: the naive matrix multiplication}
    \label{fig:mat-mult}
\end{figure}

As soon as we work on data structures bigger and more complex than the one presented before, the order of the accesses starts to matter.
For instance if we want to multiply two matrices of $8192*8192$ doubles, we can use a naive (sequential) algorithm that loops over the three matrices.
\fig{mat-mult} depicts the memory access pattern of this algorithm.
We can see that for the matrices $A$ and $C$, we loop over a whole row before going to the next column, while we go through $B$ column first.
To understand why this patterns matters and how, we have to consider the memory representation of these matrices.
As the memory address only posses one dimension, a matrix a contiguous block of memory.
Usually a matrix is represented row first, which means that $A[i][j]$ is actually $A[i*N+j]$ where $N$ is the size of a row.
This means that when we loop through a matrix row first, we scan linearly the address space, while if we do it column first, we jump of a size $N$ between two accesses.
The first access to a row of $B$ will trigger a cache miss, thus the whole cache line will be fetched.
Before we access the second element of this row, we will have to fetch one line of cache for each row of the matrix which means $8192*64=512$Kb which is often more than the size of the L2 cache.
Therefore each access to $B$ will trigger at least a L2 cache miss resulting on a lot of traffic on the bus memory and some contention.
With huge matrices, it may not even fit in the L3 cache resulting in contention in the memory bus.
The simplest way to fix this issue (although it is not the optimal algorithm for the matrix multiplication) consists in swapping the two inner loop of the algorithm.
Indeed the order of the operations does not change the results of the multiplication and this swap only change the order of the accesses concerning the matrix $B$.

To conclude, using efficiently the memory is challenging.
Indeed, due to the organization of the memory by pages and cache lines, we must consider where and how our data structures are allocated.
Furthermore the hierarchical organization of the memory and the cache must be correlated with the thread placement and data sharing.
In the end of the day, the developer must consider the machine architecture and the memory access patterns over the address space, time and threads.
Therefore visualizing the memory access pattern of an application is a great help to optimize it.

\section{Existing Tools and Methodology}
\label{sec:mem-tools}

%% From Moca Paper, rewrite
Most performance analyses are based on hardware performance counters, which are CPU
register initially designed by CPU vendors to debug their prototypes. They contain
the count of specific events such as branch prediction misses or cache misses.
Theses counters are accessible directly  through the \texttt{Perf} driver
since Linux kernel $2.6.31$ but they are CPU and vendor dependent. Thus, higher
level libraries such as \emph{PAPI}~\cite{Weaver13PAPI} and
\emph{Likwid}~\cite{Treibig10LIKWID} were developed to ease their use.
In addition, these higher level libraries are able to derivate more abstract metrics. Several works propose to analyze memory 
by looking solely at the information collected through these
libraries~\cite{Majo13(Mis)understanding,
Jiang14Understanding,Bosch00Rivet,Weyers14Visualization,Tao01Visualizing,DeRose01Hardware}.
%
Several generic tools have been designed on top of hardware performance counters
to analyze and improve parallel applications performances, such as Intel's
VTune~\cite{Reinders05VTune}, Performance Counter
Monitor~(PCM)~%\cite{Intel2012b}
, the HPCToolkit~\cite{Adhianto10HPCTOOLKIT},
and AMD's CodeAnalyst~\cite{Drongowski08introduction}.
%
Although performance counters provide information about the memory use
(bandwith, volume of data transferred \ldots),  they consider the memory as
one huge entity and do not differentiate distinct addresses or at least
distinct pages. Thus, these methods are not able to locate issues in the memory.

Some tools see the memory as a set of pages, loosing information at a finer
granularity. This approximation enable to trace memory accesses at a reduced
cost. For instance, \gls{Tabarnac}~\cite{Beniamine15TABARNAC} uses a binary
instrumentation (based on Intel's Pin~\cite{Luk05Pin}) and traps each
memory access, but it only keeps one counter per page and per threads in order to
reduce its overhead. While this approach provides a deeper insight about the
memory use than hardware performance counters, it lacks temporal information.

Tracing all the memory accesses without information loss is nearly impossible as
almost each instruction can trigger a memory access in addition to its fetch. Nevertheless, several methods
can record a \emph{detailed} memory trace with a good \emph{precision}.
%
Budanur et al.~\cite{Budanur11Memory} use an instrumentation based tool to
collect all the memory accesses. They loose \emph{precision} by doing online compression and merging accesses
into a higher level model, but this is necessary to reduce both the trace size and its overhead.
Still, on a small matrix multiplication ($48*48$, 4 threads OpenMP) they already
slow the execution down by a factor of $50$.
% sampling:
Other methods, implemented by several
tools~\cite{Lachaize12MemProf,McCurdy10Memphis,Liu14Tool,Gimenez14Dissecting},
rely on hardware sampling such as AMD's Instruction Based Sampling
(IBS)~\cite{Drongowski07Instructionbased} or Intel Precise Event Based
Sampling (PEBS)~\cite{Levinthal09Performance}.  These methods provide \emph{incomplete}
sampling: some parts of the memory can be accessed without being noticed by
the tool if none of the associated instructions are part of the sampled
instructions.  Thus, it is possible that they ignore memory areas
less frequently accessed, but in which optimization could take place.
Applications sensitive to spurious performance
degradation, such as interactive applications, could be hindered by these
unnoticed accesses, despite their low frequency.
%
% Folding mechanisms: offtopic
%
% To make things practical, these sampling mechanisms monitor what they name an events set given by an instruction type along with some predicates.
% They can monitor several events sets at the same time but the number of monitored sets is limited by the hardware capabilities (number of available
% registers). Unfortunately, the number of existing events sets that relate to the memory hierarchy is large, because of its complexity.
% This makes difficult the task of tracing all the relevant memory accesses with just a single analysis.
% One way to lessen the impact of this limitation is to run several times the
% instrumentation and use advanced methods such as
% folding~\cite{Servat15Towards} to generate a more accurate summary trace.
% Nevertheless, this makes the instrumentation cost grow accordingly.
% Moreover, writing (and sometimes) using tools that relies on hardware mechanisms
% requires a deep knowledge of the processor. As processors evolve,
% such tools are hard to maintain and can quickly become outdated.
% We regard all these limitations as too constraining for a general purpose
% memory analysis tool.

Other studies rely on hardware modifications, either actual or
simulated~\cite{Bao08HMTT,Martonosi92MemSpy}.  Although they are eventually
able to collect more \emph{precise} traces efficiently, these techniques are limited
to hardware developers. Indeed, to use these hardware extensions one has
either to obtain (or build) a prototype or to use a suitable simulator. Such
configuration is not realistic for general purpose memory analysis.

Finally, page faults interception can provide useful online information about
memory usage. Such a mechanism has been used in several existing works : in
parallel garbage collectors~\cite{Boehm91Mostly}, in memory
checkpointing~\cite{Heo05Spaceefficient} or in the domain of virtualization to
provide the hypervisor with information about the memory usage of the guest
OS~\cite{Jones06Geiger}. However, page faults only occur when caused by
predetermined events in the system (copy-on-write, paging, ...). Thus, just intercepting existing page
faults only provide an approximate view of the memory use. To improve this method,
it is also possible to fake invalid
pages at regular intervals in order to generate false
page faults~\cite{Bae12Dynamic,Diener13CommunicationBased}.  These false page
faults are just triggered during regular memory accesses, that would not have
caused a page fault if the page were not faked as invalid. The advantage is
that they create additional events for the monitoring tool to collect, thus
more \emph{precision}, but the set of faked invalid pages has to be known and
maintained by the monitoring tool.

As a final note, tools close to our proposal do not use false page faults injection and only need to store the location of memory pages and the threads that access them.
As a consequence, they require a relatively small data structure in memory for their own usage.
In this study we present \gls{Moca}, a new \emph{complete} memory trace collection system, based on page
fault interception and false page faults injection, able to capture \emph{precisely} the temporal evolution of memory accesses performed by a multithreaded
application.
To reach a satisfying \emph{precision}, our tool has to maintain in memory both the trace data and
the set of faked invalid pages. Overall, storing and exploiting efficiently these data within the kernel space and outputting them in real time to the user space
is a challenge and is the main contribution of our work.

\section{Conclusions}
\label{sec:mem-cncl}

Need global info / temporal evolution
% vim: et si sta lbr  sw=4 ts=4 spelllang=en_us
