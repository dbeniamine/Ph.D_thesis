%!TEX encoding=UTF-8 Unicode

Since a few decades, to reduce energy consumption, processor vendors builds more and more parallel computers.
At the same time, the gap between processors and memory frequency increased significantly.
To mitigate this gap, processors embed a complex hierarchical caches architecture.
Writing efficient code for such computers is a complex task.
Therefore, performance analysis has became an important step of the development of applications performing heavy computations.

Most existing performance analysis tools focuses on the point of view of the processor.
Theses tools see the main memory as a monolithic entity and thus are not able to understand how it is accessed.
However, memory is a common bottleneck in \gls{HPC}, and the pattern of memory accesses can impact significantly the performances.
There are a few tools to analyze memory performances, however theses tools are based on a coarse grain sampling.
Consequently, they focus on a small part of the execution missing the global memory behavior.
Furthermore, these coarse grain sampling are not able to collect memory accesses patterns.

In this thesis we propose two different tools to analyze the memory behavior of an application.
The first tool is designed specifically for \gls{NUMA} machines and provides some visualizations of the global sharing pattern inside each data structure between the threads.
The second one collects fine grain memory traces with temporal information.
We can visualize theses traces either with a generic trace management framework or with a programmatic exploration using R.
Furthermore we evaluate both of these tools, comparing them with state of the art memory analysis tools in terms of performances, precision and completeness.

\glsresetall
% vim: et si sta lbr  sw=4 ts=4 spelllang=en_us
